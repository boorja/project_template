\documentclass{bmcart}

%%%%%%%%%%%%%%%%%%%%%%%%%%%%%%%%%%%%%%%%%%%%%%
%%                                          %%
%% CARGA DE PAQUETES DE LATEX               %%
%%                                          %%
%%%%%%%%%%%%%%%%%%%%%%%%%%%%%%%%%%%%%%%%%%%%%%

%%% Load packages
\usepackage{amsthm,amsmath}
\usepackage{graphicx}
%\RequirePackage[numbers]{natbib}
\usepackage[hidelinks]{hyperref}
\usepackage[utf8]{inputenc} %unicode support
\usepackage{textgreek}
%\usepackage[applemac]{inputenc} %applemac support if unicode package fails
%\usepackage[latin1]{inputenc} %UNIX support if unicode package fails
\usepackage{graphicx}
\usepackage{booktabs} % Para tablas más bonitas
\usepackage{listings}
\usepackage{xcolor}
\usepackage{float}

% Configuración para mostrar código R
\lstset{
	language=R,
	basicstyle=\small\ttfamily,
	keywordstyle=\color{blue},
	commentstyle=\color{green!60!black},
	stringstyle=\color{orange},
	numbers=left,
	numberstyle=\tiny,
	stepnumber=1,
	numbersep=5pt,
	backgroundcolor=\color{gray!10},
	frame=single,
	captionpos=b,
	breaklines=true
}



%%%%%%%%%%%%%%%%%%%%%%%%%%%%%%%%%%%%%%%%%%%%%%
%%                                          %%
%% COMIENZO DEL DOCUMENTO                   %%
%%                                          %%
%%%%%%%%%%%%%%%%%%%%%%%%%%%%%%%%%%%%%%%%%%%%%%

\begin{document}
	
	\begin{frontmatter}
		
		\begin{fmbox}
			\dochead{Research}
			
			%%%%%%%%%%%%%%%%%%%%%%%%%%%%%%%%%%%%%%%%%%%%%%
			%% INTRODUCIR TITULO PROYECTO               %%
			%%%%%%%%%%%%%%%%%%%%%%%%%%%%%%%%%%%%%%%%%%%%%%
			
			\title{El Fenómeno de Raynaud a través de la Biología de Sistemas: Un Estudio Integral}
			
			%%%%%%%%%%%%%%%%%%%%%%%%%%%%%%%%%%%%%%%%%%%%%%
			%% AUTORES. METER UNA ENTRADA AUTHOR        %%
			%% POR PERSONA                              %%
			%%%%%%%%%%%%%%%%%%%%%%%%%%%%%%%%%%%%%%%%%%%%%%
			
			\author[
			addressref={aff1},
			corref={aff1},
			email={borja@uma.es}
			]{\inits{B.P.H.}\fnm{Borja} \snm{Pérez Herencia}}
			\author[
			addressref={aff1},
			email={rubenmanuel2003@gmail.com}
			]{\inits{R.M.R.C.}\fnm{Rubén Manuel} \snm{Rodríguez Chamorro}}
			\author[
			addressref={aff1},
			email={martinacs@uma.es}
			]{\inits{M.C.S.}\fnm{Martina} \snm{Cebolla Salas}}
			\author[
			addressref={aff1},
			email={emiliosancho@uma.es}
			]{\inits{E.S.C.}\fnm{Emilio} \snm{Sancho Carrera}}
			
			%%%%%%%%%%%%%%%%%%%%%%%%%%%%%%%%%%%%%%%%%%%%%%
			%% AFILIACION. NO TOCAR                     %%
			%%%%%%%%%%%%%%%%%%%%%%%%%%%%%%%%%%%%%%%%%%%%%%
			
			\address[id=aff1]{%                           % unique id
				\orgdiv{ETSI Informática},             % department, if any
				\orgname{Universidad de Málaga},          % university, etc
				\city{Málaga},                              % city
				\cny{España}                                    % country
			}
			
		\end{fmbox}% comment this for two column layout
		
		\begin{abstractbox}
			
			\begin{abstract} % abstract
				
				%%%%%%%%%%%%%%%%%%%%%%%%%%%%%%%%%%%%%%%%%%%%%%%
				%% RESUMEN BREVE DE NO MAS DE 100 PALABRAS   %%
				%%%%%%%%%%%%%%%%%%%%%%%%%%%%%%%%%%%%%%%%%%%%%%%	
				
				El Fenómeno de Raynaud (FR) es un trastorno vasomotor cuya base molecular permanece incompletamente caracterizada. Mediante un enfoque de biología de sistemas, integramos datos fenotípicos de la Human Phenotype Ontology con redes de interacción proteína-proteína de STRINGdb. Construimos una red de 39 genes y 67 interacciones, identificando siete módulos funcionales (modularidad $Q=0.55$). El análisis reveló el eje de interferonopatías tipo I como componente central, con \textit{IRF5} como hub principal e \textit{IFIH1} y \textit{ADAR} como bottlenecks críticos. Nuestros resultados sugieren que el FR representa una manifestación periférica de desregulación inmune sistémica, proponiendo nuevas dianas terapéuticas.
				
			\end{abstract}
			
			%%%%%%%%%%%%%%%%%%%%%%%%%%%%%%%%%%%%%%%%%%%%%%
			%% PALABRAS CLAVE DEL PROYECTO              %%
			%%%%%%%%%%%%%%%%%%%%%%%%%%%%%%%%%%%%%%%%%%%%%%
			
			\begin{keyword}
				\kwd{Raynaud}
				\kwd{biología de sistemas}
				\kwd{interferón}
				\kwd{redes PPI}
				\kwd{HPO}
				\kwd{autoinmunidad}
				\kwd{análisis de redes}
				\kwd{enriquecimiento funcional}
				\kwd{vasoespasmo}
			\end{keyword}
			
			
		\end{abstractbox}
		
	\end{frontmatter}
	
	
	
	%%%%%%%%%%%%%%%%%%%%%%%%%%%%%%%%%
	%% COMIENZO DEL DOCUMENTO REAL %%
	%%%%%%%%%%%%%%%%%%%%%%%%%%%%%%%%%
	
	\section{Introducción}

Millones de personas experimentan episodios de dedos pálidos, cianóticos y dolorosos con el frío, un fenómeno frecuente pero a menudo infradiagnosticado. El fenómeno de Raynaud (FR) constituye un trastorno vasospástico multifactorial caracterizado por la constricción transitoria, recurrente y reversible de los vasos sanguíneos periféricos \cite{Nawaz2022}.  Clínicamente, se manifiesta mediante un patrón trifásico de decoloración en los dedos: palidez inicial por isquemia, cianosis por falta de oxigenación y, finalmente, eritema durante la reperfusión. Este trastorno afecta alrededor del 5\% de la población general y muestra una marcada predisposición femenina, con una relación de hasta 9:1 \cite{Medscape2024, Musa2023}.  

La patofisiología del fenómeno de Raynaud involucra una compleja interacción de mecanismos vasculares, neurales e intravasculares que alteran el equilibrio entre vasoconstricción y vasodilatación \cite{Herrick2005}. A nivel molecular, se han identificado tres mecanismos principales: anomalías en el flujo sanguíneo, constricción vascular y respuestas neurogénicas. El sistema nervioso simpático desempeña un papel crucial mediante la liberación de norepinefrina y neuropéptidos vasoconstrictores que actúan sobre el músculo liso arteriolar. De particular relevancia es el papel de los receptores adrenérgicos $\alpha$2C, cuya translocación desde el retículo endoplasmático hacia la membrana celular en respuesta al frío contribuye significativamente al vasoespasmo exagerado \cite{Fardoun2016, Flavahan2008}.La patofisiología del fenómeno de Raynaud refleja un desequilibrio entre vasoconstricción y vasodilatación en la microcirculación digital, donde convergen alteraciones endoteliales, del músculo liso vascular y de la modulación simpática \cite{Herrick2005}. A nivel celular, este proceso se inicia con una disfunción endotelial que compromete la producción de vasodilatadores clave como el óxido nítrico y la prostaciclina, mientras incrementa la liberación de endotelina-1 \cite{Flavahan2008, Blann1993}. Este desequilibrio bioquímico no permanece confinado al endotelio, las células del músculo liso vascular responden a estas señales alteradas con una hiperreactividad vasoconstrictora exagerada y proliferación intimal progresiva \cite{Fardoun2016}. A nivel molecular, el sistema nervioso simpático perpetúa y amplifica este ciclo mediante la liberación sostenida de norepinefrina y neuropéptidos vasoconstrictores. El mecanismo molecular central involucra a los receptores adrenérgicos $\alpha$2C, que en respuesta al frío experimentan una translocación desde el retículo endoplasmático hacia la membrana celular contribuyendo así al vasoespasmo característico del trastorno \cite{Fardoun2016, Flavahan2008}.

Recientes avances en genómica han revelado importantes hallazgos sobre la base genética del fenómeno de Raynaud. Un estudio de asociación del genoma completo (GWAS) de gran escala identificó por primera vez genes causales robustamente asociados con el FR, destacando particularmente \textbf{ADRA2A} e \textbf{IRX1} como genes de susceptibilidad \cite{Hartmann2023}. El gen \textbf{ADRA2A} codifica el receptor adrenérgico $\alpha$2A para la adrenalina, un receptor de estrés clásico que causa la contracción de pequeños vasos sanguíneos. Por otro lado, \textbf{IRX1} es un factor de transcripción que puede regular la capacidad de los vasos sanguíneos para dilatarse, y su sobreproducción puede activar genes que impiden la relajación normal de los vasos constrictos \cite{ofLondon2023}.

La aplicación de enfoques de biología de sistemas ofrece un marco ideal para descifrar esta complejidad. La Ontología de Fenotipos Humanos (HPO) proporciona una descripción estandarizada de las anomalías fenotípicas, permitiendo un análisis computacional robusto que vincula los síntomas clínicos a sus bases genéticas \cite{Khler2021, Robinson2008}. Complementariamente, el análisis de redes de interacción proteína-proteína, utilizando bases de datos como STRING, permite modelar y visualizar las relaciones funcionales entre los genes asociados al fenotipo \cite{Szklarczyk2025}. Esta aproximación integradora es clave para identificar módulos funcionales y vías de señalización alteradas, revelando así potenciales dianas terapéuticas \cite{Consortium}.

En este contexto, el presente trabajo tiene como objetivo integrar el conocimiento fenotípico estandarizado del HPO con análisis de redes de interacción proteica para elucidar la arquitectura molecular del fenómeno de Raynaud. Mediante este enfoque de biología de sistemas, se busca no solo validar la implicación de genes conocidos, sino también identificar nuevas vías biológicas y módulos funcionales que contribuyen a la patogénesis de este complejo trastorno vasospástico, sentando así las bases para futuras investigaciones terapéuticas \cite{Naylor2010, Fischer2025}.

	\section{Materiales y métodos}

\subsection{Materiales}

\subsubsection{Bases de datos biológicas}
\begin{itemize}
	\item \textbf{Human Phenotype Ontology (HPO):} Fuente principal para la obtención de genes asociados a fenotipos clínicos. HPO utiliza un vocabulario estandarizado que permite una vinculación sistemática y reproducible entre las características de una enfermedad y su base genética \cite{Groza2023, Khler2021}.
	
	\item \textbf{Base de Datos STRING:} Se empleó para construir la red de interacciones. Esta es una base de datos integral que recopila y pondera interacciones proteína-proteína (PPI) a partir de múltiples fuentes de evidencia (experimental, computacional, etc.), asignando a cada una un puntaje de confianza \cite{Szklarczyk2025}.
\end{itemize}

\subsubsection{Software y Paquetes de Análisis}
Todos los análisis se realizaron en el entorno de programación \textbf{R} (v4.5.2 o superior), utilizando los siguientes paquetes:
\begin{itemize}
	\item \texttt{httr} (vx.x.x): Para la comunicación con la API de HPO.
	\item \texttt{jsonlite} (vx.x.x): Para el procesamiento de los datos en formato JSON obtenidos de la API.
	\item \texttt{STRINGdb} (vx.x.x): Para consultar la base de datos STRING y construir la red de PPI \cite{Szklarczyk2019}.
	\item \texttt{igraph} (vx.x.x): Herramienta central para la modelación, visualización y cálculo de propiedades topológicas de la red \cite{Csardi2006}.
	\item \texttt{clusterProfiler} (vx.x.x): Paquete de Bioconductor para la ejecución de análisis de enriquecimiento funcional \cite{Wu2021}.
\end{itemize}

\subsubsection{Algoritmos y Enfoques Estadísticos}
\begin{itemize}
	\item \textbf{Algoritmo de Detección de Comunidades de Louvain:} Es un método heurístico utilizado para identificar la estructura modular en redes complejas. El algoritmo optimiza iterativamente una métrica de modularidad ($Q$), que cuantifica la densidad de las conexiones dentro de las comunidades en comparación con las conexiones entre ellas. La modularidad se define como:
	\begin{equation}
		Q = \frac{1}{2m} \sum_{i,j} \left[ A_{ij} - \frac{k_i k_j}{2m} \right] \delta(c_i, c_j)
		\label{eq:louvain_modularity}
	\end{equation}
	Donde:
	\begin{itemize}
		\item $A_{ij}$ es un elemento de la matriz de adyacencia, que vale 1 si los nodos $i$ y $j$ están conectados y 0 en caso contrario.
		\item $k_i$ y $k_j$ son los grados (número de conexiones) de los nodos $i$ y $j$.
		\item $m$ es el número total de aristas en la red.
		\item $\frac{k_i k_j}{2m}$ representa la probabilidad esperada de que exista una arista entre $i$ y $j$ en una red aleatoria con la misma distribución de grados.
		\item $c_i$ y $c_j$ son las comunidades a las que pertenecen los nodos $i$ y $j$.
		\item $\delta(c_i, c_j)$ es la función delta de Kronecker, que vale 1 si los nodos están en la misma comunidad ($c_i = c_j$) y 0 en caso contrario.
	\end{itemize}
	El algoritmo busca la partición de la red que maximiza el valor de $Q$, revelando subgrupos de nodos densamente conectados que se postula que comparten funciones biológicas.
	
	\item \textbf{Análisis de Sobrerrepresentación (ORA):} Es un enfoque estadístico para determinar si un conjunto de genes de interés está significativamente enriquecido en funciones o vías biológicas predefinidas. El método se basa en la prueba hipergeométrica para calcular el p-valor, que es la probabilidad de observar una superposición igual o mayor a la encontrada por puro azar. La fórmula es:
	\begin{equation}
		P(X \ge k) = \sum_{i=k}^{\min(n,K)} \frac{\binom{K}{i} \binom{N-K}{n-i}}{\binom{N}{n}}
		\label{eq:ora_hypergeometric}
	\end{equation}
	Donde:
	\begin{itemize}
		\item $N$ es el número total de genes en el genoma de fondo (background).
		\item $K$ es el número total de genes asociados al término funcional en estudio dentro del fondo.
		\item $n$ es el número de genes en el conjunto de interés (ej. genes sobreexpresados).
		\item $k$ es el número de genes en el conjunto de interés que también están asociados al término funcional.
	\end{itemize}
	Un p-valor bajo sugiere que la sobrerrepresentación observada no es casual, sino biológicamente significativa.
	
	\item \textbf{Corrección de Benjamini-Hochberg (BH):} Al realizar miles de pruebas estadísticas simultáneamente (una por cada término funcional), la probabilidad de obtener falsos positivos (errores de tipo I) se incrementa. El método de Benjamini-Hochberg controla la Tasa de Falso Descubrimiento (FDR), que es la proporción esperada de descubrimientos incorrectos. El procedimiento ordena los p-valores de menor a mayor ($p_{(1)} \le p_{(2)} \le \dots \le p_{(m)}$) y encuentra el mayor $k$ tal que:
	\begin{equation}
		p_{(k)} \le \frac{k}{m} \alpha
		\label{eq:bh_fdr}
	\end{equation}
	Donde:
	\begin{itemize}
		\item $p_{(k)}$ es el k-ésimo p-valor más pequeño de la lista ordenada.
		\item $m$ es el número total de pruebas realizadas (el número total de términos GO evaluados).
		\item $\alpha$ es el nivel de FDR que se desea controlar (comúnmente 0.05).
	\end{itemize}
	Todas las hipótesis nulas correspondientes a los p-valores $p_{(1)}, \dots, p_{(k)}$ se rechazan, considerándose significativas. Este ajuste es menos conservador que la corrección de Bonferroni y es ampliamente utilizado en genómica \cite{Benjamini1995}.
\end{itemize}

\subsection{Métodos}

\subsubsection{Adquisición de Datos y Construcción de la Red}
El conjunto inicial de genes se obtuvo mediante una consulta a la API de la \textbf{Human Phenotype Ontology (HPO)} utilizando el término "Raynaud phenomenon" (ID: \textbf{HP:0030881}). La lista de símbolos de genes resultante se utilizó como entrada para construir una red de interacción proteína-proteína (PPI) para \textit{Homo sapiens} a partir de la base de datos \textbf{STRING}. Se aplicó un filtro de alta confianza, reteniendo únicamente las interacciones con un \textbf{puntaje combinado superior a 0.800}. La red final se representó como un grafo no dirigido.

\subsubsection{Análisis Estructural y de Comunidades}
Las propiedades topológicas de la red fueron analizadas con el paquete \texttt{igraph}. Se calcularon métricas globales (número de nodos y aristas, densidad) y locales (grado promedio, centralidad de cercanía). Posteriormente, se aplicó el \textbf{algoritmo de Louvain} para particionar la red en módulos funcionales basándose en su estructura de conectividad.


\subsubsection{Análisis de Enriquecimiento Funcional}
Cada módulo identificado en el paso anterior fue sometido a un \textbf{Análisis de Sobrerrepresentación (ORA)} utilizando el paquete \texttt{clusterProfiler}. Se evaluó el enriquecimiento de términos de la ontología de \textbf{Proceso Biológico (BP)} de Gene Ontology (GO). Los p-valores resultantes se ajustaron mediante el método de \textbf{Benjamini-Hochberg}, y se consideraron como estadísticamente significativos aquellos términos con un \textbf{p-valor ajustado inferior a 0.05}.



	\section{Resultados}

\subsection{Obtención de los genes asociados al fenotipo (HPO)}

El estudio comenzó con la identificación de la base genética asociada al fenotipo clínico. A partir del término \textbf{Raynaud Phenomenon (HPO:0030880)}, se realizó una recuperación programática mediante la API oficial de la \textit{Human Phenotype Ontology}. Tras el procesamiento de la respuesta JSON y la depuración de duplicados, se consolidó un conjunto inicial de \textbf{56 genes} candidatos. Este conjunto constituyó el punto de partida biológico para las fases posteriores del análisis topológico.
	
\section{Discusión}

\subsection{Resumen Interpretativo}
El presente estudio ha aplicado un enfoque de biología de sistemas para analizar la complejidad molecular del Fenómeno de Raynaud. A través del análisis topológico y funcional de la red PPI, hemos demostrado que la patología no emerge de genes aislados, sino de la interacción coordinada de cuatro módulos funcionales distintos. Los hallazgos indican que la arquitectura de la enfermedad se sostiene sobre un eje principal de señalización de interferón y activación inmune innata, el cual está intrínsecamente conectado, a través de nodos clave, con procesos de inestabilidad nuclear y senescencia celular.
	\section{Conclusiones}

El presente trabajo demuestra el valor de la biología de sistemas como herramienta para reinterpretar el Fenómeno de Raynaud desde una perspectiva integrada. Al combinar fenotipos clínicos y redes de interacción proteica, hemos podido observar la patología no como la consecuencia de alteraciones aisladas, sino como el resultado emergente de la interacción entre múltiples procesos celulares y moleculares. Este enfoque ha permitido identificar patrones globales que trascienden la visión tradicional centrada exclusivamente en el tono vascular.

Los resultados obtenidos destacan que la fisiopatología del fenotipo no puede comprenderse plenamente sin considerar la contribución de mecanismos inmunológicos, de mantenimiento nuclear y de respuesta al daño celular. El análisis de red sugiere que la activación inflamatoria, la senescencia y la vigilancia inmunitaria conforman un eje funcional común que podría desempeñar un papel más relevante de lo previamente reconocido en este trastorno. Esta perspectiva sistémica invita a ampliar el marco conceptual del Raynaud hacia una visión más compleja, en la que convergen factores estructurales, inmunológicos y de señalización intracelular.
	
	
	%%%%%%%%%%%%%%%%%%%%%%%%%%%%%%%%%%%%%%%%%%%%%%
	%% OTRA INFORMACIÓN                         %%
	%%%%%%%%%%%%%%%%%%%%%%%%%%%%%%%%%%%%%%%%%%%%%%
	
	\begin{backmatter}
		
		\section*{Abreviaciones}
		\textbf{FR}: Fenómeno de Raynaud; 
		\textbf{PPI}: Interacción proteína-proteína (\textit{Protein-Protein Interaction}); 
		\textbf{HPO}: Human Phenotype Ontology; 
		\textbf{GO}: Gene Ontology; 
		\textbf{BP}: Proceso Biológico (\textit{Biological Process}); 
		\textbf{IFN}: Interferón; 
		\textbf{AGS}: Síndrome de Aicardi-Goutières; 
		\textbf{LES}: Lupus Eritematoso Sistémico; 
		\textbf{ORA}: Análisis de Sobrerrepresentación (\textit{Over-Representation Analysis}); 
		\textbf{FDR}: Tasa de Falso Descubrimiento (\textit{False Discovery Rate}); 
		\textbf{BH}: Benjamini-Hochberg; 
		\textbf{GWAS}: Estudio de Asociación del Genoma Completo (\textit{Genome-Wide Association Study}); 
		\textbf{SNP}: Polimorfismo de Nucleótido Único (\textit{Single Nucleotide Polymorphism}); 
		\textbf{TLR}: Receptor Tipo Toll (\textit{Toll-Like Receptor}); 
		\textbf{cGAS-STING}: Vía de señalización cGAS-STING; 
		\textbf{MDA5}: Proteína 5 Asociada a Diferenciación de Melanoma (codificada por \textit{IFIH1}); 
		\textbf{dsRNA}: RNA de doble cadena (\textit{double-stranded RNA}); 
		\textbf{ADCC}: Citotoxicidad Celular Dependiente de Anticuerpos (\textit{Antibody-Dependent Cell-mediated Cytotoxicity}); 
		\textbf{HGPS}: Síndrome de Progeria de Hutchinson-Gilford.
		
		\section*{Disponibilidad de datos y materiales}
		El código fuente, los datos y los materiales utilizados en este estudio están disponibles en el repositorio de GitHub: \href{https://github.com/boorja/project_template}{https://github.com/boorja/project\_template}
		
		\section*{Contribución de los autores}
		\textbf{B.P.H.}: Desarrollo del código en R (construcción de la red PPI, análisis topológico con igraph, detección de comunidades Louvain), scripts de automatización y entorno, redacción de materiales y métodos; 
		\textbf{R.M.R.C.}: Desarrollo del código en R (consultas a la API de HPO, mapeo con STRINGdb, análisis de enriquecimiento funcional con clusterProfiler), redacción de resultados, visualización de la red con ggraph; 
		\textbf{M.C.S.}: Implementación de funciones de visualización (gráficos de enriquecimiento, exportación de figuras), redacción de la introducción, elaboración de la discusión, revisión bibliográfica; 
		\textbf{E.S.C.}: Desarrollo de funciones auxiliares, redacción de las conclusiones, elaboración de tablas, revisión bibliográfica y edición final del manuscrito.
		Todos los autores contribuyeron al diseño del estudio, revisaron y aprobaron la versión final del manuscrito. 
		
		
		%%%%%%%%%%%%%%%%%%%%%%%%%%%%%%%%%%%%%%%%%%%%%%%%%%%%%%%%%%%%%%%%%%%%%%%%%%%%%%%%%%%%%%%%
		%% BIBLIOGRAFIA: no teneis que tocar nada, solo sustituir el archivo bibliography.bib %%
		%% por el que hayais generado vosotros                                                %%
		%%%%%%%%%%%%%%%%%%%%%%%%%%%%%%%%%%%%%%%%%%%%%%%%%%%%%%%%%%%%%%%%%%%%%%%%%%%%%%%%%%%%%%%%
		
		\bibliographystyle{bmc-mathphys} % Style BST file (bmc-mathphys, vancouver, spbasic).
		\bibliography{bibliography}      % Bibliography file (usually '*.bib' )
		
	\end{backmatter}
\end{document}
