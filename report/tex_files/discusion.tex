
\section{Discusión}

El presente análisis de redes revela que el Fenómeno de Raynaud posee una arquitectura molecular considerablemente más compleja que la de un simple trastorno vasomotor. La red PPI construida a partir de 39 genes asociados al fenotipo HPO:0030880 exhibe una modularidad elevada ($Q = 0.55$), indicando una organización funcional en siete comunidades biológicamente coherentes. Este hallazgo sugiere que la patogénesis del FR involucra la interacción coordinada de múltiples ejes moleculares, donde la señalización de interferón tipo I, la inmunidad innata y la regulación epigenética convergen en un fenotipo clínico común.

El hallazgo más relevante de nuestro análisis es la identificación del Cluster 3 como módulo central de la red, dominado por genes asociados a interferonopatías tipo I. Las interferonopatías constituyen un grupo de enfermedades monogénicas caracterizadas por activación constitutiva de la vía del interferón tipo I, descritas por primera vez como entidad nosológica por Crow en 2011 \cite{Crow2011, Lee-Kirsch2015}. Este cluster agrupa a \textit{TREX1}, \textit{RNASEH2A/B/C}, \textit{SAMHD1}, \textit{ADAR} y \textit{STING1}, todos ellos implicados en el metabolismo de ácidos nucleicos y la regulación de la respuesta a interferón \cite{Rice2007, Crow2015}. La presencia de estos genes establece un vínculo molecular directo entre el Fenómeno de Raynaud y el síndrome de Aicardi-Goutières (AGS), una enfermedad caracterizada por una producción constitutiva de IFN-$\alpha$ \cite{Rice2007}. En condiciones fisiológicas, \textit{TREX1} actúa como exonucleasa 3'$\rightarrow$5' que degrada el DNA citosólico, previniendo la activación aberrante de la vía cGAS-STING \cite{Du2025, Stetson2006}. El descubrimiento de cGAS como sensor citosólico de DNA por Sun et al. en 2013 revolucionó nuestra comprensión de la inmunidad innata, demostrando que la enzima produce cGAMP que activa STING y desencadena la producción de interferones tipo I \cite{Sun2013, Ablasser2013}. Las mutaciones con pérdida de función en \textit{TREX1} conducen a la acumulación de DNA endógeno que activa esta cascada, generando un estado proinflamatorio crónico \cite{An2018, Gao2023, Rodero2016}.

La posición topológica de \textit{ADAR} como bottleneck ($k=5$, betweenness $= 0.239$) merece especial atención. ADAR1 edita dsRNA endógeno mediante deaminación de adenosina a inosina, previniendo su reconocimiento por sensores inmunes como MDA5 (\textit{IFIH1}) \cite{Chung2018, Ahmad2018}. La pérdida de esta función permite que secuencias Alu invertidas formen dúplex de RNA que activan MDA5, conectando funcionalmente el Cluster 3 con el Cluster 1 a través de \textit{IFIH1} \cite{Ahmad2018}. Esta conexión topológica refleja una realidad biológica profunda: la convergencia de múltiples sensores de ácidos nucleicos hacia una vía efectora común.

El Cluster 1 representa precisamente el brazo efector de esta respuesta inmune innata, con genes como \textit{TLR7}, \textit{IFIH1} (MDA5), \textit{FCGR2A/B}, \textit{ITGAM} y \textit{CR2}. La centralidad de intermediación excepcionalmente alta de \textit{IFIH1} ($k=4$, betweenness $= 0.306$) lo posiciona como el principal puente entre la detección de ácidos nucleicos aberrantes y la activación de la respuesta inflamatoria (Figura \ref{fig:Red_Final}). MDA5 es un sensor citosólico de dsRNA largo que, al activarse, induce la producción de IFN tipo I a través de MAVS \cite{Rice2020, Nombel2021}. Mutaciones de ganancia de función en \textit{IFIH1} causan un espectro de interferonopatías que incluyen AGS y síndrome de Singleton-Merten, frecuentemente acompañados de manifestaciones vasculares periféricas similares al Raynaud \cite{Rice2020}. Nuestros datos sugieren que \textit{IFIH1} actúa como un ``amplificador'' que traduce señales de estrés celular en respuestas inflamatorias sistémicas. El enriquecimiento en señalización de receptores Fc$\gamma$ (GO:0002768) y citotoxicidad celular dependiente de anticuerpos apunta además a un componente humoral significativo, conectando este módulo con procesos de depósito de inmunocomplejos observados en LES y esclerosis sistémica \cite{Herrick2012, Ingegnoli2022}.

La arquitectura de la red sitúa a \textit{IRF5} como hub principal ($k=10$, betweenness $= 0.344$), liderando el Cluster 4 que representa el eje de activación de células B y respuesta inflamatoria. IRF5 es un factor de transcripción crítico para la amplificación de citoquinas proinflamatorias, y su asociación con enfermedades autoinmunes ha sido replicada en múltiples poblaciones étnicas \cite{Kozyrev2007, Sigurdsson2005, Graham2006}. Los polimorfismos en \textit{IRF5} que aumentan su expresión o actividad se asocian con niveles séricos elevados de IFN-$\alpha$ en pacientes con LES, proporcionando un mecanismo directo que conecta variación genética con fenotipo inmunológico \cite{Niewold2007, Tsuchiya2010}. La inhibición farmacológica de IRF5 ha demostrado eficacia en modelos murinos de lupus, reduciendo la producción de autoanticuerpos y la nefritis \cite{Ban2021}. Nuestros resultados posicionan a IRF5 no solo como un gen de susceptibilidad, sino como un integrador topológico que conecta la señalización de receptores tipo Toll (a través de \textit{IRAK1}) con la activación de células B (\textit{BANK1}, \textit{BLK}, \textit{PTPN22}). El enriquecimiento en producción de IL-6 (GO:0032635) y señalización del receptor de células B (GO:0050853) es coherente con el papel establecido de variantes de \textit{PTPN22} (R620W) en la susceptibilidad a múltiples enfermedades autoinmunes \cite{Rosetti2019}. Estudios recientes de Mayes et al. han confirmado la asociación de múltiples loci inmunológicos, incluyendo \textit{IRF5} y \textit{STAT4}, con esclerosis sistémica, enfermedad donde el FR es frecuentemente el primer síntoma \cite{Mayes2014}.

El Cluster 5 agrupa genes fundamentales para la tolerancia inmunológica, incluyendo el hub secundario \textit{IL10} ($k=9$, betweenness $= 0.168$), \textit{HLA-DRB1}, \textit{PDCD1} (PD-1), \textit{CTLA4} y componentes del complemento \textit{C4A/C4B}. Este módulo representa el contrapeso regulador de la red. IL-10 es la principal citoquina antiinflamatoria, y su centralidad sugiere un papel homeostático crítico: la disfunción en este nodo podría facilitar la transición desde una predisposición genética hacia una autoinmunidad clínicamente manifiesta. Los checkpoints inmunes PD-1 y CTLA-4, enriquecidos en regulación negativa de activación de células T (GO:0050868), previenen la autorreactividad; su importancia se evidencia dramáticamente en los fenómenos autoinmunes observados durante la inmunoterapia oncológica con inhibidores de checkpoint \cite{Fasano2023}. La presencia de \textit{C4A/C4B} vincula este cluster con el aclaramiento deficiente de debris apoptótico, un mecanismo patogénico establecido en LES donde la deficiencia de complemento predispone a autoinmunidad \cite{Herrick2012}.

El Cluster 6 presenta una firma funcional inesperada dominada por regulación epigenética (GO:0040029), formación de heterocromatina (GO:0031507) y procesamiento de miRNA (GO:1903799). Este módulo incluye \textit{LMNA}, \textit{ZMPSTE24}, \textit{LBR} y \textit{MECP2}, genes asociados a la integridad de la envoltura nuclear y la organización de la cromatina. Las mutaciones en \textit{LMNA} causan un espectro de laminopatías que incluyen cardiomiopatías, distrofias musculares y síndromes progeroides \cite{Fardoun2016}. Notablemente, el síndrome de progeria de Hutchinson-Gilford (HGPS), causado por mutaciones en \textit{LMNA}, cursa con disfunción endotelial severa y enfermedad vascular acelerada. La enzima ZMPSTE24 procesa la prelamina A; su deficiencia causa displasia mandibuloacral y dermopathy restrictive, condiciones con afectación vascular prominente. La conexión topológica entre este cluster y el resto de la red a través de \textit{IRAK1} ($k=4$, betweenness $= 0.171$) sugiere un modelo ``de adentro hacia afuera'' donde la inestabilidad nuclear podría liberar DNA al citosol, activando la vía cGAS-STING y conectando con el eje de interferonopatías del Cluster 3 \cite{Du2023}.

La integración de estos hallazgos permite proponer un modelo patogénico coherente (Figura \ref{fig:Red_Final}). En este modelo, alteraciones en genes del Cluster 6 (laminopatías, inestabilidad nuclear) o del Cluster 3 (deficiencia en nucleasas como TREX1/RNaseH2) conducen a la acumulación de ácidos nucleicos citosólicos. Estos son detectados por sensores innatos (cGAS-STING, MDA5/\textit{IFIH1}, TLR7) que activan la producción sostenida de IFN tipo I. La señal de interferón amplifica la respuesta inmune a través de IRF5 (Cluster 4), promoviendo la activación de células B y la producción de autoanticuerpos. La disfunción del eje regulador IL-10/CTLA4/PD-1 (Cluster 5) permite la perpetuación de esta respuesta. El resultado final es un estado inflamatorio crónico con disfunción endotelial, depósito de inmunocomplejos y vasoespasmo periférico---las manifestaciones clínicas del Fenómeno de Raynaud. Este modelo reconcilia la aparente discordancia entre los estudios GWAS, que priorizan genes vasculares como \textit{ADRA2A} \cite{Hartmann2023}, y nuestros hallazgos centrados en inmunidad. Proponemos que los mecanismos vasculares representan el ``efector final'', mientras que la desregulación inmune constituye el ``motor'' que perpetúa la enfermedad.

La identificación de bottlenecks específicos sugiere potenciales dianas terapéuticas. Los inhibidores de cGAS han mostrado eficacia en modelos murinos de interferonopatía por deficiencia de TREX1 \cite{An2018}. Los inhibidores de JAK, que bloquean la señalización downstream del receptor de interferón, están siendo evaluados en dermatomiositis anti-MDA5 positiva, una condición con solapamiento fenotípico con el FR \cite{Nombel2021}. Finalmente, la inhibición farmacológica de IRF5 representa una estrategia emergente para LES y potencialmente para el FR asociado a autoinmunidad \cite{Song2020, Ban2021}.

Este trabajo presenta limitaciones inherentes a los análisis basados en redes PPI. Los datos de STRINGdb agregan evidencias de múltiples tejidos y contextos experimentales, sin capturar la especificidad del endotelio vascular o el músculo liso. El algoritmo de Louvain asigna cada gen a un único cluster, simplificando la multifuncionalidad de proteínas como \textit{TREX1}, relevante tanto para reparación de DNA como para inmunidad innata. Además, el umbral de confianza utilizado (score $> 700$) excluye interacciones de menor evidencia que podrían ser biológicamente relevantes.

\subsection{Perspectivas futuras}

Los hallazgos presentados abren múltiples líneas de investigación. Sería fundamental validar experimentalmente si la inducción de senescencia en células endoteliales activa de forma autónoma el módulo de interferón, como sugiere nuestro modelo. La integración con datos de transcriptómica de célula única de pacientes permitiría transformar esta red estática en un modelo dinámico capaz de predecir el impacto de intervenciones farmacológicas sobre nodos específicos. Finalmente, estudios prospectivos podrían evaluar si la firma de interferón tipo I sirve como biomarcador para identificar pacientes con FR primario en riesgo de progresión a enfermedad autoinmune sistémica.

