
\section{Discusión}

\subsection{Resumen Interpretativo}
El presente estudio ha aplicado un enfoque de biología de sistemas para analizar la complejidad molecular del Fenómeno de Raynaud. A través del análisis topológico y funcional de la red PPI, hemos demostrado que la patología no emerge de genes aislados, sino de la interacción coordinada de cuatro módulos funcionales distintos. Los hallazgos indican que la arquitectura de la enfermedad se sostiene sobre un eje principal de señalización de interferón y activación inmune innata, el cual está intrínsecamente conectado, a través de nodos clave, con procesos de inestabilidad nuclear y senescencia celular.