
\section{Discusión}

\subsection{Resumen Interpretativo}
El presente estudio ha aplicado un enfoque de biología de sistemas para analizar la complejidad molecular del Fenómeno de Raynaud. A través del análisis topológico y funcional de la red PPI, hemos demostrado que la patología no emerge de genes aislados, sino de la interacción coordinada de cuatro módulos funcionales distintos. Los hallazgos indican que la arquitectura de la enfermedad se sostiene sobre un eje principal de señalización de interferón y activación inmune innata, el cual está intrínsecamente conectado, a través de nodos clave, con procesos de inestabilidad nuclear y senescencia celular.

\subsection{Interpretación Biológica de la Red}
\vspace{6pt}
\textbf{La firma de Interferón como eje patogénico central.}\\[6pt]
La identificación del Clúster 3, dominado casi exclusivamente por la señalización de interferón de tipo I, constituye uno de los hallazgos más relevantes. La presencia de \textit{IFIH1} (MDA5) y \textit{STING1} como nodos centrales en este módulo respalda la hipótesis de que el Fenómeno de Raynaud comparte mecanismos fisiopatológicos con las interferonopatías autoinflamatorias. El hecho de que \textit{IFIH1} posea el mayor valor de intermediación (\textit{betweenness}) de toda la red sugiere que actúa como un controlador del flujo de información, traduciendo señales de estrés celular (como la presencia de ácidos nucleicos virales o propios) en una respuesta inflamatoria sistémica que podría desencadenar el vasoespasmo.\cite{Herrick2012} \cite{Ingegnoli2022}

\vspace{6pt}
\textbf{Conexión mecanicista entre daño al ADN y autoinmunidad.}\\[6pt]
El Clúster 4 introduce una perspectiva novedosa al vincular la "senescencia celular" con la red inmunológica general. La literatura previa ha establecido que mutaciones en \textit{TREX1} (presente en la interfaz de los clústeres 3 y 4) provocan la acumulación de ADN en el citosol, activando la vía cGAS-STING. Nuestros resultados integran computacionalmente estas observaciones: la inestabilidad de la lámina nuclear (representada por \textit{LMNA} y \textit{ZMPSTE24} en el Clúster 4) podría ser la fuente primaria de daño celular que libera ADN propio, activando crónicamente el módulo de interferón (Clúster 3) sin necesidad de una infección viral externa. Esto sugiere un modelo de patogénesis "de adentro hacia afuera", donde el daño tisular precede a la activación inmune.\cite{deAlmeidaTervi2024}

A diferencia de lo expuesto en la introducción, donde se destacó el rol de la disfunción endotelial y la hiperactividad adrenérgica ($ADRA2A$, $NOS3$), nuestra red priorizó mecanismos de daño estructural e inflamación. Esto sugiere que, si bien el vasoespasmo es el síntoma clínico, la maquinaria molecular subyacente en los genes asociados por HPO está dominada por efectores inmunológicos, relegando a los reguladores del tono vascular a un segundo plano o a una dependencia de la señalización inflamatoria previa.\cite{Chotani2000} \cite{Smolina2018}

\vspace{6pt}
\textbf{El rol compensatorio de IL10.}\\[6pt]
Es notable la alta centralidad de \textit{IL10} en el Clúster 2. Siendo una potente citocina antiinflamatoria, su posición estratégica conectando múltiples clústeres sugiere que la red intenta mantener un estado de homeostasis mediante mecanismos de retroalimentación negativa. La pérdida de funcionalidad en este nodo específico podría ser el evento precipitante que permite la transición de una predisposición genética a una manifestación clínica autoinmune florida.





