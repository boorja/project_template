
\section{Discusión}


El análisis basado en biología de sistemas revela que el Fenómeno de Raynaud no se explica por la acción aislada de genes individuales, sino por la interacción coordinada de cuatro módulos funcionales dentro de la red PPI. La arquitectura global muestra que la señalización de interferón de tipo I y la activación inmune innata constituyen el eje molecular dominante, estrechamente interconectado con procesos de inestabilidad nuclear y senescencia celular. Este patrón sugiere que la patología presenta una organización más inmunológica e inflamatoria que puramente vascular, lo que redefine su marco conceptual.


Nuestros resultados apuntan a un mecanismo “de adentro hacia afuera” en el que el daño celular antecede y condiciona la respuesta clínica. La presencia de proteínas asociadas a la lámina nuclear, como \textit{LMNA} y \textit{ZMPSTE24}, dentro del Clúster 4, indica que la inestabilidad estructural del núcleo podría generar la liberación aberrante de ADN al citosol. Este ADN, unido a alteraciones en \textit{TREX1}, activaría la vía cGAS-STING, un mecanismo ampliamente descrito en interferonopatías autoinflamatorias. Una vez activado, este eje cGAS-STING convergería sobre nodos de alta intermediación como \textit{IFIH1} (MDA5) y \textit{STING1}, amplificando la producción de interferón tipo I y desencadenando una respuesta inflamatoria sistémica. Proponemos, por tanto, que el vasoespasmo característico del Fenómeno de Raynaud sería un evento secundario generado por esta señalización inflamatoria persistente, y no necesariamente por alteraciones primarias del tono vascular.

En este marco mecanístico, los módulos funcionales de la red aportan un contexto adicional. El Clúster 3 destaca como el núcleo patogénico central, al estar dominado casi exclusivamente por la señalización de interferón de tipo I. El valor excepcionalmente alto de \textit{betweenness} de \textit{IFIH1} posiciona a este gen como un regulador del flujo de información inflamatoria, capaz de traducir señales de estrés celular en respuestas clínicas. Por su parte, el Clúster 4 vincula de manera novedosa la senescencia celular con la inmunidad innata. La conexión topológica entre \textit{LMNA}, \textit{ZMPSTE24} y \textit{TREX1} sugiere que el daño al ADN y la alteración de la envoltura nuclear no son eventos aislados, sino pasos iniciales dentro de la cascada inmunopatológica.

Finalmente, la red muestra un componente de regulación compensatoria a través del Clúster 2, donde \textit{IL10} adopta un papel modulador clave. Su alta centralidad apunta a un intento de mantener la homeostasis mediante retroalimentación antiinflamatoria. La disfunción en este nodo podría facilitar la transición desde una predisposición genética a una manifestación autoinmune plenamente establecida, lo que refuerza la importancia de considerar tanto los ejes proinflamatorios como los mecanismos amortiguadores en el modelo global de la enfermedad.


Al comparar nuestros hallazgos con el reciente estudio GWAS de Hartmann et al.\cite{Hartmann2023}, observamos diferencias importantes entre susceptibilidad genética y mecanismo efector. Mientras el GWAS prioriza genes asociados a regulación vascular como \textit{ADRA2A}, nuestro análisis topológico destaca nodos inmunológicos como \textit{IRF5}, \textit{IFIH1} y \textit{STING1}. Esto refuerza la idea de que, aunque el disparador inicial pueda ser vascular, la perpetuación de la enfermedad depende de mecanismos inflamatorios sistémicos.

Asimismo, la asociación entre \textit{IRF5}, \textit{TNFAIP3} y la esclerosis sistémica registrada en estudios previos se ve complementada por nuestros datos, que sitúan estos genes dentro de una red más amplia que incluye vías de daño nuclear. De esta manera, aportamos un modelo más unificado que integra inmunidad innata, senescencia y respuesta inflamatoria crónica.

\subsection{Limitaciones del Estudio}
Este trabajo presenta limitaciones inherentes al uso de redes PPI basadas en datos agregados de STRINGdb, que no capturan la especificidad de tejido ni el contexto fisiológico del endotelio o del músculo liso vascular. Además, el algoritmo de Louvain fuerza una partición rígida que asigna cada gen a un único clúster, lo que podría simplificar en exceso la multifuncionalidad de proteínas como \textit{TREX1}, relevantes tanto para la reparación del ADN como para la inmunidad innata.


\subsection{Perspectivas Futuras}
Los módulos identificados ofrecen múltiples líneas de avance. Sería fundamental validar si la inducción de senescencia en células endoteliales es capaz de activar de forma autónoma el módulo de interferón (Clúster 3), tal como sugiere nuestro modelo computacional. La integración con transcriptómica de célula única de pacientes permitiría transformar esta red estática en un sistema dinámico, capaz de predecir el impacto de intervenir farmacológicamente nodos clave como \textit{IFIH1} o \textit{STING1}. Este enfoque podría abrir la puerta a nuevas terapias dirigidas basadas en la modulación de vías inmunológicas profundas en lugar de receptores vasomotores periféricos.



