
\section{Discusión}

\subsection{Resumen Interpretativo}
El presente estudio ha aplicado un enfoque de biología de sistemas para analizar la complejidad molecular del Fenómeno de Raynaud. A través del análisis topológico y funcional de la red PPI, hemos demostrado que la patología no emerge de genes aislados, sino de la interacción coordinada de cuatro módulos funcionales distintos. Los hallazgos indican que la arquitectura de la enfermedad se sostiene sobre un eje principal de señalización de interferón y activación inmune innata, el cual está intrínsecamente conectado, a través de nodos clave, con procesos de inestabilidad nuclear y senescencia celular.

\subsection{Interpretación Biológica de la Red}
\vspace{6pt}
\textbf{La firma de Interferón como eje patogénico central.}\\[6pt]
La identificación del Clúster 3, dominado casi exclusivamente por la señalización de interferón de tipo I, constituye uno de los hallazgos más relevantes. La presencia de \textit{IFIH1} (MDA5) y \textit{STING1} como nodos centrales en este módulo respalda la hipótesis de que el Fenómeno de Raynaud comparte mecanismos fisiopatológicos con las interferonopatías autoinflamatorias. El hecho de que \textit{IFIH1} posea el mayor valor de intermediación (\textit{betweenness}) de toda la red sugiere que actúa como un controlador del flujo de información, traduciendo señales de estrés celular (como la presencia de ácidos nucleicos virales o propios) en una respuesta inflamatoria sistémica que podría desencadenar el vasoespasmo.\cite{Herrick2012} \cite{Ingegnoli2022}