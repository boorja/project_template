
\section{Discusión}

\subsection{Síntesis General de los Hallazgos}
El análisis basado en biología de sistemas revela que el Fenómeno de Raynaud no se explica por la acción aislada de genes individuales, sino por la interacción coordinada de cuatro módulos funcionales dentro de la red PPI. La arquitectura global muestra que la señalización de interferón de tipo I y la activación inmune innata constituyen el eje molecular dominante, estrechamente interconectado con procesos de inestabilidad nuclear y senescencia celular. Este patrón sugiere que la patología presenta una organización más inmunológica e inflamatoria que puramente vascular, lo que redefine su marco conceptual.

\subsection{Interpretación Mecanística y Funcional de la Red}

Nuestros resultados apuntan a un mecanismo “de adentro hacia afuera” en el que el daño celular antecede y condiciona la respuesta clínica. La presencia de proteínas asociadas a la lámina nuclear, como \textit{LMNA} y \textit{ZMPSTE24}, dentro del Clúster 4, indica que la inestabilidad estructural del núcleo podría generar la liberación aberrante de ADN al citosol. Este ADN, unido a alteraciones en \textit{TREX1}, activaría la vía cGAS-STING, un mecanismo ampliamente descrito en interferonopatías autoinflamatorias. Una vez activado, este eje cGAS-STING convergería sobre nodos de alta intermediación como \textit{IFIH1} (MDA5) y \textit{STING1}, amplificando la producción de interferón tipo I y desencadenando una respuesta inflamatoria sistémica. Proponemos, por tanto, que el vasoespasmo característico del Fenómeno de Raynaud sería un evento secundario generado por esta señalización inflamatoria persistente, y no necesariamente por alteraciones primarias del tono vascular.

En este marco mecanístico, los módulos funcionales de la red aportan un contexto adicional. El Clúster 3 destaca como el núcleo patogénico central, al estar dominado casi exclusivamente por la señalización de interferón de tipo I. El valor excepcionalmente alto de \textit{betweenness} de \textit{IFIH1} posiciona a este gen como un regulador del flujo de información inflamatoria, capaz de traducir señales de estrés celular en respuestas clínicas. Por su parte, el Clúster 4 vincula de manera novedosa la senescencia celular con la inmunidad innata. La conexión topológica entre \textit{LMNA}, \textit{ZMPSTE24} y \textit{TREX1} sugiere que el daño al ADN y la alteración de la envoltura nuclear no son eventos aislados, sino pasos iniciales dentro de la cascada inmunopatológica.

Finalmente, la red muestra un componente de regulación compensatoria a través del Clúster 2, donde \textit{IL10} adopta un papel modulador clave. Su alta centralidad apunta a un intento de mantener la homeostasis mediante retroalimentación antiinflamatoria. La disfunción en este nodo podría facilitar la transición desde una predisposición genética a una manifestación autoinmune plenamente establecida, lo que refuerza la importancia de considerar tanto los ejes proinflamatorios como los mecanismos amortiguadores en el modelo global de la enfermedad.

\subsection{Comparación con la Literatura}
Al contrastar nuestros hallazgos con el reciente estudio GWAS de gran escala \cite{Hartmann2023} observamos una distinción biológica relevante. Mientras que dicho estudio identificó a \textit{ADRA2A} e \textit{IRX1} como los principales loci de susceptibilidad genética, estos genes no emergieron como nodos centrales (\textit{hubs}) en nuestro análisis topológico. En su lugar, nuestra red priorizó a reguladores inmunes como \textit{IRF5} e \textit{IFIH1}.

Esta discrepancia valida nuestro enfoque computacional al ofrecer una visión complementaria: mientras que el GWAS captura el riesgo genético inicial (la causa), nuestra red de interacciones proteínas ilustra la maquinaria efectora (el mecanismo), sugiriendo que, independientemente del detonante adrenérgico inicial, la patología se ejecuta y perpetúa a través de una red inflamatoria masiva. Asimismo, nuestros resultados refuerzan asociaciones previas de \textit{IRF5} y \textit{TNFAIP3} con la esclerosis sistémica, pero aportamos un valor añadido al demostrar su conexión topológica directa con la inestabilidad nuclear (\textit{LMNA}), una vía de daño menos explorada que la regulación de citocinas clásica.


\subsection{Limitaciones del Estudio}
Es necesario reconocer ciertas limitaciones en nuestro enfoque. En primer lugar, la red PPI se construyó utilizando la base de datos STRINGdb, que aglomera interacciones teóricas y experimentales globales. Estas interacciones no son específicas de tejido, por lo que la red podría no reflejar fielmente el microambiente endotelial o de músculo liso vascular donde ocurre el vasoespasmo. En segundo lugar, el algoritmo de Louvain fuerza una partición estricta de la red (cada gen pertenece a un solo clúster), lo que podría simplificar en exceso la realidad biológica de proteínas pleiotrópicas como \textit{TREX1}, que funcionalmente opera en la interfaz entre la reparación del ADN y la inmunidad innata.


\subsection{Perspectivas Futuras}
Los módulos identificados abren nuevas líneas de investigación. Sería importante verificar experimentalmente si la inducción de senescencia en células endoteliales es suficiente para activar el módulo de interferón (Clúster 3) que se identificó mediante análisis computacional. Además, combinar estos datos estáticos con perfiles de expresión génica de pacientes (como transcriptómica de célula única) podría ayudar a ajustar las conexiones de la red y convertir este modelo topológico en uno dinámico. Esto permitiría predecir la respuesta a fármacos que bloqueen nodos clave, como \textit{IFIH1} o \textit{STING1}.



