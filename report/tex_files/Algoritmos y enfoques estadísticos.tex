\subsubsection{Algoritmos y Enfoques Estadísticos}
\begin{itemize}
	\item \textbf{Algoritmo de Detección de Comunidades de Louvain:} Es un método heurístico utilizado para identificar la estructura modular en redes complejas. El algoritmo optimiza iterativamente una métrica de "modularidad", que cuantifica la densidad de las conexiones dentro de las comunidades en comparación con las conexiones entre ellas. Su aplicación permite particionar la red en subgrupos de nodos densamente conectados, que se postula que comparten funciones biológicas.
	
	\item \textbf{Análisis de Sobrerrepresentación (ORA):} Es un enfoque estadístico utilizado para determinar si un conjunto de genes de interés está significativamente enriquecido en funciones o vías biológicas predefinidas. El método evalúa, para cada término funcional, si la proporción de genes asociados a él dentro del conjunto de interés es mayor de lo que se esperaría por azar, basándose en la frecuencia de ese término en un conjunto de fondo (el genoma completo).
	
	\item \textbf{Corrección de Benjamini-Hochberg (BH):} Al realizar miles de pruebas estadísticas simultáneamente (una por cada término GO), aumenta la probabilidad de obtener falsos positivos. El método de Benjamini-Hochberg es un procedimiento estándar para controlar la tasa de falso descubrimiento (FDR), ajustando los p-valores para reducir la proporción de resultados significativos que en realidad son fruto del azar \cite{Benjamini1995}.
\end{itemize}

