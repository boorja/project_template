\section{Conclusiones}

El presente trabajo demuestra el valor de la biología de sistemas como herramienta para reinterpretar el Fenómeno de Raynaud desde una perspectiva integrada. Al combinar fenotipos clínicos y redes de interacción proteica, hemos podido observar la patología no como la consecuencia de alteraciones aisladas, sino como el resultado emergente de la interacción entre múltiples procesos celulares y moleculares. Este enfoque ha permitido identificar patrones globales que trascienden la visión tradicional centrada exclusivamente en el tono vascular.