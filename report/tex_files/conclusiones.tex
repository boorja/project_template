\section{Conclusiones}

El presente estudio ha logrado deconstruir la arquitectura molecular del Fenómeno de Raynaud mediante un enfoque de medicina de redes, revelando una organización biológica que trasciende la visión clásica vasocéntrica. Nuestra integración de fenotipos (HPO) e interactoma (PPI) permite establecer tres conclusiones fundamentales:

\begin{enumerate}
	\item \textbf{Redefinición Patogénica:} La topología de la red demuestra que, si bien la manifestación clínica es vascular, la maquinaria molecular subyacente es predominantemente inmunológica. La identificación del \textit{Cluster 3} (Firma de Interferón) y la alta centralidad de \textit{IFIH1} sitúan al Raynaud en el espectro de las interferonopatías, sugiriendo que el vasoespasmo es un evento secundario a una respuesta inflamatoria sistémica.
	
	\item \textbf{Mecanismo "Inside-Out":} Hemos identificado un vínculo mecanicista inédito entre la inestabilidad de la lámina nuclear (\textit{Cluster 4}, \textit{LMNA}) y la activación inmune innata. Esto apoya un modelo donde la senescencia celular y el daño al ADN actúan como señales de peligro (DAMPs) que activan crónicamente la vía del interferón, independientemente de infecciones externas.
	
	\item \textbf{Validación frente a Genética Clásica:} A diferencia de los estudios GWAS que priorizan genes de riesgo vascular (\textit{ADRA2A}), nuestro enfoque de red ha priorizado los "cuellos de botella" funcionales de la enfermedad (\textit{IRF5}, \textit{STING1}). Esto implica que, para el tratamiento clínico, podría ser más efectivo bloquear estos nodos efectores de la inflamación que intentar modular receptores adrenérgicos periféricos.
\end{enumerate}

En definitiva, proponemos que el Fenómeno de Raynaud no debe considerarse únicamente como un trastorno de hiperreactividad vascular local, sino como una patología emergente de la interacción entre el envejecimiento celular y la desregulación inmune innata. Estos hallazgos abren la puerta al reposicionamiento de fármacos inhibidores de la vía JAK/STAT como potenciales terapias dirigidas.