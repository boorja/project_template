\section{Conclusiones}

El presente trabajo demuestra el valor de la biología de sistemas como herramienta para reinterpretar el Fenómeno de Raynaud desde una perspectiva integrada. Al combinar fenotipos clínicos y redes de interacción proteica, hemos podido observar la patología no como la consecuencia de alteraciones aisladas, sino como el resultado emergente de la interacción entre múltiples procesos celulares y moleculares. Este enfoque ha permitido identificar patrones globales que trascienden la visión tradicional centrada exclusivamente en el tono vascular.

Los resultados obtenidos destacan que la fisiopatología del fenotipo no puede comprenderse plenamente sin considerar la contribución de mecanismos inmunológicos, de mantenimiento nuclear y de respuesta al daño celular. El análisis de red sugiere que la activación inflamatoria, la senescencia y la vigilancia inmunitaria conforman un eje funcional común que podría desempeñar un papel más relevante de lo previamente reconocido en este trastorno. Esta perspectiva sistémica invita a ampliar el marco conceptual del Raynaud hacia una visión más compleja, en la que convergen factores estructurales, inmunológicos y de señalización intracelular.