\section{Conclusiones}

El presente trabajo demuestra la utilidad de la biología de sistemas para reinterpretar el Fenómeno de Raynaud más allá de su caracterización clásica como trastorno vasomotor periférico. Mediante la integración de datos fenotípicos de la Human Phenotype Ontology con redes de interacción proteína-proteína de STRINGdb, hemos construido un modelo que revela la arquitectura molecular subyacente a esta condición, identificando siete módulos funcionales interconectados con una modularidad significativa ($Q = 0.55$).

El hallazgo central de este estudio es la identificación del eje de las interferonopatías tipo I como componente nuclear de la red. La presencia de genes como \textit{TREX1}, \textit{RNASEH2A/B/C}, \textit{SAMHD1} y \textit{ADAR}---todos ellos causantes de síndrome de Aicardi-Goutières cuando están mutados---establece un vínculo molecular directo entre el Fenómeno de Raynaud y la activación constitutiva de la vía del interferón. Este nexo, mediado por la cascada cGAS-STING y los sensores de ácidos nucleicos como MDA5 (\textit{IFIH1}), proporciona una base mecanística para comprender por qué el FR aparece frecuentemente como manifestación temprana de enfermedades autoinmunes sistémicas.

El análisis topológico identificó a \textit{IRF5} como hub principal de la red, conectando la señalización de receptores tipo Toll con la activación de células B y la producción de citoquinas proinflamatorias. La convergencia de múltiples genes de susceptibilidad a lupus eritematoso sistémico y esclerosis sistémica en nuestra red refuerza la noción de que el FR representa, en muchos casos, la manifestación periférica de una desregulación inmune subyacente. Los bottlenecks identificados---\textit{IFIH1}, \textit{ADAR} e \textit{IRAK1}---emergen como potenciales dianas terapéuticas cuya modulación podría interrumpir la cascada patogénica en puntos estratégicos.

Finalmente, la identificación de un módulo de regulación epigenética y estructura nuclear, centrado en \textit{LMNA} y genes asociados a laminopatías, sugiere un modelo ``de adentro hacia afuera'' donde la inestabilidad nuclear podría contribuir a la activación del eje de interferón. Este hallazgo abre nuevas perspectivas para comprender la conexión entre envejecimiento vascular, senescencia celular y fenómenos vasomotores.

En conjunto, nuestros resultados proponen un cambio de paradigma: el Fenómeno de Raynaud no debe conceptualizarse únicamente como un trastorno del tono vascular, sino como la manifestación clínica de una red inmuno-inflamatoria donde la señalización de interferón tipo I ocupa un papel central. Esta visión integradora, fundamentada en la biología de sistemas, sienta las bases para el desarrollo de biomarcadores predictivos y estrategias terapéuticas dirigidas que trasciendan el tratamiento sintomático actual.