\section{Resultados}

\subsection{Obtención de los genes asociados al fenotipo (HPO)}

El estudio comenzó con la identificación de la base genética asociada al fenotipo clínico. A partir del término \textbf{Raynaud Phenomenon (HPO:0030880)}, se realizó una recuperación programática mediante la API oficial de la \textit{Human Phenotype Ontology}. Tras el procesamiento de la respuesta JSON y la depuración de duplicados, se consolidó un conjunto inicial de \textbf{56 genes} candidatos. Este conjunto constituyó el punto de partida biológico para las fases posteriores del análisis topológico.

\subsection{Construcción de la red de interacción con STRINGdb}

Los genes recuperados se mapearon a sus identificadores de proteína en la base de datos \textbf{STRINGdb} (versión 12.0) \cite{Szklarczyk2025}. Durante este proceso, aproximadamente un \textbf{3\%} de los genes no pudo ser mapeado debido a la ausencia de correspondencia directa o anotación en la base de datos de referencia.

Utilizando los identificadores válidos, se construyó la red de interacción proteína-proteína (PPI) aplicando un umbral de confianza (\textit{combined score}) de 400. Este criterio permitió recuperar interacciones de confianza media-alta, abarcando tanto asociaciones físicas directas como relaciones funcionales. La red cruda fue sometida a un proceso de limpieza topológica para eliminar lazos propios (\textit{self-loops}) y aristas redundantes. El resultado final fue una red conectada compuesta por **54 nodos** y **228 aristas**.