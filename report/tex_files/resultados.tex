\section{Resultados}

\subsection{Obtención de los genes asociados al fenotipo (HPO)}

El estudio comenzó con la identificación de la base genética asociada al fenotipo clínico. A partir del término \textbf{Raynaud Phenomenon (HPO:0030880)}, se realizó una recuperación programática mediante la API oficial de la \textit{Human Phenotype Ontology}. Tras el procesamiento de la respuesta JSON y la depuración de duplicados, se consolidó un conjunto inicial de genes candidatos que constituyó el punto de partida biológico para las fases posteriores del análisis topológico.

\subsection{Construcción y Topología de la Red}

Los genes recuperados se mapearon a sus identificadores de proteína en la base de datos \textbf{STRINGdb} (versión 12.0) \cite{Szklarczyk2025}. Utilizando un umbral de confianza (\textit{combined score}) de 400, se construyó la red de interacción proteína-proteína (PPI). Tras el proceso de filtrado topológico para eliminar nodos desconectados, la red final quedó constituida por \textbf{39 nodos} y \textbf{67 aristas} (Figura \ref{fig:Network_Blobs}).

El análisis de las métricas globales (Tabla \ref{tab:global_stats}) revela una red con una densidad baja ($0.09$), característica de redes biológicas dispersas, pero con una estructura comunitaria definida (Modularidad $Q = 0.5487$). El coeficiente de agrupamiento medio (\textit{Clustering Coefficient}) de $0.4269$ sugiere la presencia de nichos funcionales locales bien conectados, a pesar de la baja densidad global.

\begin{table}[h!]
	\centering
	\caption{Estadísticas Globales de la Red}
	\label{tab:global_stats}
	\begin{tabular}{lc}
		\toprule
		\textbf{Métrica} & \textbf{Valor} \\
		\midrule
		Nodos Totales & 39 \\
		Aristas Totales & 67 \\
		Densidad de la Red & 0.090 \\
		Grado Medio & 3.44 \\
		Longitud de Camino Medio & 3.68 \\
		Diámetro & 10 \\
		Coeficiente de Clustering & 0.427 \\
		Modularidad (Q) & 0.549 \\
		\bottomrule
	\end{tabular}
\end{table}

\begin{figure}[h!]
	\centering
	\includegraphics[width=0.8\textwidth]{figures/Clusters_Blobs.png}
	\caption{Visualización de la red PPI final (39 nodos). Los colores representan los diferentes clusters funcionales detectados por el algoritmo de modularidad.}
	\label{fig:Network_Blobs}
\end{figure}

\subsection{Análisis de Centralidad: Identificación de Hubs y Cuellos de Botella}

El análisis de centralidad permitió identificar los nodos críticos que sostienen la arquitectura de la red (Tabla \ref{tab:nodes_info}). El factor de transcripción \textbf{\textit{IRF5}} emergió como el principal \textit{hub} de la red, presentando el mayor grado ($k=10$) y la mayor centralidad de intermediación (\textit{Betweenness} $= 0.343$). Esto posiciona a \textit{IRF5} como un regulador maestro, capaz de influir en múltiples módulos funcionales y actuar como un cuello de botella en el flujo de información biológica.

Otros nodos destacados incluyen a \textbf{\textit{IL10}} ($k=9$) y \textbf{\textit{HLA-DRB1}}, fundamentales en la regulación inmune adaptativa. Es notable el papel de \textbf{\textit{ADAR}} ($k=3$), que a pesar de tener un grado moderado, exhibe la segunda mayor intermediación ($0.239$), sugiriendo un rol crucial como "puente" o conector entre distintos clusters funcionales, probablemente vinculando la detección de ácidos nucleicos con la respuesta inmune global.

\begin{table}[h!]
	\centering
	\caption{Top 5 Genes por Centralidad (Grado e Intermediación)}
	\label{tab:nodes_info}
	\begin{tabular}{lcccc}
		\toprule
		\textbf{Gen} & \textbf{Cluster} & \textbf{Grado} & \textbf{Betweenness} & \textbf{Función Principal} \\
		\midrule
		\textit{IRF5} & 4 & 10 & 0.344 & Activación Cel. B / Regulación \\
		\textit{IL10} & 5 & 9 & 0.168 & Anti-inflamatorio \\
		\textit{PTPN22} & 4 & 6 & 0.024 & Señalización Cel. T/B \\
		\textit{HLA-DRB1} & 5 & 5 & 0.104 & Presentación de antígenos \\
		\textit{ADAR} & 3 & 5 & 0.239 & Edición de RNA / Inmunidad viral \\
		\bottomrule
	\end{tabular}
\end{table}

\subsection{Análisis Modular y Enriquecimiento Funcional}

La descomposición de la red mediante el algoritmo de modularidad reveló distintas comunidades funcionales. El análisis de enriquecimiento de ontología génica (GO) permitió caracterizar biológicamente estos clusters:

\begin{itemize}
	\item \textbf{Cluster 1: Sistema del Complemento y Fagocitosis.} Este módulo, que incluye genes como \textit{ITGAM} y componentes del complemento (\textit{C1QA, C1QB}), mostró un enriquecimiento significativo en la regulación de la activación del complemento y procesos de fagocitosis (Figura \ref{fig:Enrichment_C1}).
	
	\item \textbf{Cluster 3: Respuesta a Interferón Tipo I.} Definido por la presencia de genes como \textit{RNASEH2A/B/C}, \textit{TREX1} y \textit{ADAR}. El enriquecimiento confirma una fuerte asociación con la señalización de interferón alfa y la respuesta celular ante ácidos nucleicos exógenos o endógenos, vinculando este cluster con mecanismos de autoinmunidad tipo interferonopatía (Figura \ref{fig:Enrichment_C3}).
	
	\item \textbf{Cluster 4: Activación de Células B.} Liderado por el hub \textit{IRF5} junto con \textit{BANK1} y \textit{PTPN22}, este grupo está enriquecido en procesos de activación de linfocitos B y transducción de señales de respuesta inmune, sugiriendo un componente humoral en la patología (Figura \ref{fig:Enrichment_C4}).
	
	\item \textbf{Cluster 6: Organización de la Envoltura Nuclear.} A diferencia de los módulos inmunológicos, este cluster mostró una asociación específica con la organización de la envoltura nuclear, probablemente dirigido por genes estructurales como \textit{LMNA}. Esto sugiere alteraciones en la integridad nuclear funcionalmente distintas, pero topológicamente conectadas a la respuesta inmune (Figura \ref{fig:Enrichment_C6}).
\end{itemize}

\begin{figure}[h!]
	\centering
	\begin{minipage}{0.48\textwidth}
		\centering
		\includegraphics[width=\linewidth]{figures/Enrichment_Cluster_1.png}
		\caption{Cluster 1: Regulación del Complemento.}
		\label{fig:Enrichment_C1}
	\end{minipage}\hfill
	\begin{minipage}{0.48\textwidth}
		\centering
		\includegraphics[width=\linewidth]{figures/Enrichment_Cluster_3.png}
		\caption{Cluster 3: Señalización Interferón Tipo I.}
		\label{fig:Enrichment_C3}
	\end{minipage}
\end{figure}

\begin{figure}[h!]
	\centering
	\begin{minipage}{0.48\textwidth}
		\centering
		\includegraphics[width=\linewidth]{figures/Enrichment_Cluster_4.png}
		\caption{Cluster 4: Activación de Células B.}
		\label{fig:Enrichment_C4}
	\end{minipage}\hfill
	\begin{minipage}{0.48\textwidth}
		\centering
		\includegraphics[width=\linewidth]{figures/Enrichment_Cluster_6.png}
		\caption{Cluster 6: Organización Nuclear.}
		\label{fig:Enrichment_C6}
	\end{minipage}
\end{figure}