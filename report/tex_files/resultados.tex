\section{Resultados}

\subsection{Obtención de los genes asociados al fenotipo (HPO)}

El estudio comenzó con la identificación de la base genética asociada al fenotipo clínico. A partir del término \textbf{Raynaud Phenomenon (HPO:0030880)}, se realizó una recuperación programática mediante la API oficial de la \textit{Human Phenotype Ontology}. Tras el procesamiento de la respuesta JSON y la depuración de duplicados, se consolidó un conjunto inicial de \textbf{56 genes} candidatos. Este conjunto constituyó el punto de partida biológico para las fases posteriores del análisis topológico.

\subsection{Construcción de la red de interacción con STRINGdb}

Los genes recuperados se mapearon a sus identificadores de proteína en la base de datos \textbf{STRINGdb} (versión 12.0) \cite{Szklarczyk2025}. Durante este proceso, aproximadamente un \textbf{3\%} de los genes no pudo ser mapeado debido a la ausencia de correspondencia directa o anotación en la base de datos de referencia.

Utilizando los identificadores válidos, se construyó la red de interacción proteína-proteína (PPI) aplicando un umbral de confianza (\textit{combined score}) de 400. Este criterio permitió recuperar interacciones de confianza media-alta, abarcando tanto asociaciones físicas directas como relaciones funcionales. La red cruda fue sometida a un proceso de limpieza topológica para eliminar lazos propios (\textit{self-loops}) y aristas redundantes. El resultado final fue una red conectada compuesta por **54 nodos** y **228 aristas**.


\subsection{Conversión a igraph y análisis topológico}

La red se transformó en un objeto \texttt{igraph} para profundizar en su estructura matemática. Las propiedades globales calculadas (Tabla \ref{tab:propiedades-red}) revelan una densidad de 0.1593. Este valor, junto con el grado medio de 8.44, indica que la red no es una estructura dispersa, sino un sistema cohesivo donde las proteínas tienden a formar grupos de interacción funcional significativa.

\begin{table}[htbp]
	\centering
	\caption{Propiedades topológicas globales de la red de Raynaud.}
	\label{tab:propiedades-red}
	\vspace{4pt}
	\renewcommand{\arraystretch}{1.25}
	\begin{tabular}{lc}
		\hline
		\textbf{Propiedad}            & \textbf{Valor} \\ \hline
		Número total de nodos         & 54             \\ 
		Número total de aristas       & 228            \\ 
		Grado medio                   & 8.44           \\ 
		Densidad de la red            & 0.1593         \\ 
		Diámetro de la red            & 5              \\ \hline
	\end{tabular}
\end{table}

Además de las métricas globales, se evaluó la importancia individual de los nodos mediante medidas de centralidad. El análisis de grado (\textit{degree}) y de intermediación (\textit{betweenness}) permitió identificar a los actores principales del sistema (Tabla \ref{tab:top_nodes}). 

Se observó que el gen \textbf{IRF5} actúa como el principal \textit{hub} de la red por conectividad directa. Sin embargo, el gen \textbf{IFIH1}, pese a tener menor grado, mostró el mayor valor de intermediación (0.178), lo que sugiere que actúa como un "cuello de botella" o puente crítico para el flujo de información entre diferentes módulos funcionales.




\begin{table}[htbp]
	\centering
	\caption{Genes más relevantes según métricas de centralidad.}
	\label{tab:top_nodes}
	\vspace{4pt}
	\renewcommand{\arraystretch}{1.2}
	\begin{tabular}{lccc}
		\hline
		\textbf{Gen} & \textbf{Cluster} & \textbf{Grado (Degree)} & \textbf{Intermediación (Betweenness)} \\ \hline
		IRF5    & 1 & 22 & 0.0768 \\
		ITGAM   & 2 & 21 & 0.0726 \\
		IL10    & 2 & 19 & 0.1171 \\
		IFIH1   & 3 & 19 & \textbf{0.1785} \\
		TLR7    & 2 & 18 & 0.0468 \\
		TNFAIP3 & 1 & 18 & 0.0307 \\ \hline
	\end{tabular}
\end{table}

\subsection{Visualización de la red}

Para examinar la organización espacial de estas interacciones, se generaron visualizaciones utilizando el algoritmo de disposición Fruchterman–Reingold, que optimiza la posición de los nodos basándose en fuerzas de atracción y repulsión (Figura \ref{fig:Red_Raynaud_Premium}). En dicha visualización, el tamaño de los nodos se escaló proporcionalmente a su grado, permitiendo apreciar visualmente cómo los \textit{hubs} identificados (IRF5, ITGAM) ocupan posiciones centrales y articulan la estructura de la red.

\begin{figure}
	\centering
	\includegraphics[width=0.9\linewidth]{figures/Red_Raynaud.png}
	\caption{Visualización de la red de interacción proteína-proteína. El tamaño de los nodos es proporcional a su conectividad.}
	\label{fig:Red_Raynaud_Premium}
\end{figure}

\begin{figure}[h!]
	\centering
	\includegraphics[width=0.8\linewidth]{figures/Clusters_Blobs.png}
	\caption{Comunidades detectadas por el algoritmo de Louvain. Las áreas sombreadas agrupan nodos densamente conectados.}
	\label{fig:Clusters_Blobs_Igraph}
\end{figure}

\clearpage


\\






\subsection{Detección de Comunidades (Clustering)}

Con el objetivo de definir módulos funcionales discretos, se aplicó el algoritmo de \textbf{Louvain} \cite{Csardi2006}. Este método particionó la red en \textbf{9 clústeres distintos}, optimizando la densidad de enlaces intra-grupo frente a la inter-grupo.

El análisis arrojó una modularidad global de \textbf{Q = 0.3358}, confirmando una estructura comunitaria bien definida. Aunque se detectaron 9 grupos, la distribución de tamaños (16, 16, 10, 6, 1, 1, 1, 2, 1) indica que la funcionalidad biológica reside principalmente en los cuatro primeros clústeres, mientras que el resto son componentes periféricos o aislados.

El siguiente fragmento de código ilustra la implementación del algoritmo utilizado para esta partición:

\begin{lstlisting}[caption=Detección de comunidades mediante algoritmo Louvain en R]
	# Deteccion de comunidades (Algoritmo Louvain)
	set.seed(123)
	comunidades <- cluster_louvain(g)
	V(g)$cluster <- membership(comunidades)
	
	# Evaluacion de la calidad de la particion
	Q_val <- modularity(comunidades) 
	# Resultado obtenido: Q = 0.3358
\end{lstlisting}

La Figura \ref{fig:Clusters_Blobs_Igraph} muestra la topología de estas comunidades, destacando la separación entre el núcleo denso de los clústeres 1, 2 y 3, y la posición más periférica del clúster 4. 



\subsection{Enriquecimiento funcional (GO – Biological Process)}

Para descifrar la "personalidad biológica" de cada comunidad detectada, se realizó un análisis de enriquecimiento funcional (ORA) sobre los términos de Gene Ontology (Biological Process), aplicando corrección por Benjamini–Hochberg \cite{Benjamini1995}.

El análisis estratificado reveló perfiles funcionales altamente específicos para los cuatro clústeres principales:

\paragraph{} El \textbf{Cluster 1 (Activación Inmune)} agrupa genes como \textit{RNF125} e \textit{IRAK1}. El análisis (Figura \ref{fig:Enrichment_C1}) mostró un enriquecimiento significativo en la “activación de la vía de señalización inmune innata”, lo que sugiere que este conjunto de genes actúa como motor de arranque de la respuesta inflamatoria.

\paragraph{} El \textbf{Cluster 2 (Respuesta Leucocitaria)} contiene a \textit{IL10} y diversos genes HLA. Se asoció principalmente con la “inmunidad mediada por leucocitos” y la “regulación del proceso efector inmune”, indicando un papel relevante en la presentación de antígenos y en la comunicación celular (Figura \ref{fig:Enrichment_C2}).

\paragraph{} El \textbf{Cluster 3 (Firma de Interferón)} constituye el hallazgo más distintivo. Tal como se observa en la Figura \ref{fig:Enrichment_C3}, los genes de este grupo (\textit{IFIH1}, \textit{STING1}) están casi exclusivamente dedicados a la “respuesta a interferón de tipo I” y a la “defensa antiviral”, conformando una firma molecular fuertemente especializada.

\paragraph{} Por último, el \textbf{Cluster 4 (Senescencia Nuclear)} se diferencia de los anteriores por su enriquecimiento en procesos de “senescencia celular” y “organización de la envoltura nuclear”, dirigidos por genes como \textit{LMNA} y \textit{ZMPSTE24} (Figura \ref{fig:Enrichment_C4}).
	
	\begin{figure}[h!]
	\centering
	\begin{minipage}{0.48\textwidth}
		\centering
		\includegraphics[width=\linewidth]{figures/Enrichment_Cluster_1.png}
		\caption{Análisis GO del Cluster 1: Activación Inmune.}
		\label{fig:Enrichment_C1}
	\end{minipage}\hfill
	\begin{minipage}{0.48\textwidth}
		\centering
		\includegraphics[width=\linewidth]{figures/Enrichment_Cluster_2.png}
		\caption{Análisis GO del Cluster 2: Respuesta Leucocitaria.}
		\label{fig:Enrichment_C2}
	\end{minipage}
	\end{figure}
	
	\begin{figure}[h!]
	\centering
	\begin{minipage}{0.48\textwidth}
		\centering
		\includegraphics[width=\linewidth]{figures/Enrichment_Cluster_3.png}
		\caption{Análisis GO del Cluster 3: Respuesta a Interferón.}
		\label{fig:Enrichment_C3}
	\end{minipage}\hfill
	\begin{minipage}{0.48\textwidth}
		\centering
		\includegraphics[width=\linewidth]{figures/Enrichment_Cluster_4.png}
		\caption{Análisis GO del Cluster 4: Senescencia y Núcleo.}
		\label{fig:Enrichment_C4}
	\end{minipage}
\end{figure}
	
	

