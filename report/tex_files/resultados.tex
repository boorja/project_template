\section{Resultados}

\subsection{Obtención de los genes asociados al fenotipo (HPO)}

El estudio comenzó con la identificación de la base genética asociada al fenotipo clínico. A partir del término \textbf{Raynaud Phenomenon (HPO:0030880)}, se realizó una recuperación programática mediante la API oficial de la \textit{Human Phenotype Ontology}. Tras el procesamiento de la respuesta JSON y la depuración de duplicados, se consolidó un conjunto inicial de \textbf{56 genes} candidatos. Este conjunto constituyó el punto de partida biológico para las fases posteriores del análisis topológico.

\subsection{Construcción de la red de interacción con STRINGdb}

Los genes recuperados se mapearon a sus identificadores de proteína en la base de datos \textbf{STRINGdb} (versión 12.0) \cite{Szklarczyk2025}. Durante este proceso, aproximadamente un \textbf{3\%} de los genes no pudo ser mapeado debido a la ausencia de correspondencia directa o anotación en la base de datos de referencia.

Utilizando los identificadores válidos, se construyó la red de interacción proteína-proteína (PPI) aplicando un umbral de confianza (\textit{combined score}) de 400. Este criterio permitió recuperar interacciones de confianza media-alta, abarcando tanto asociaciones físicas directas como relaciones funcionales. La red cruda fue sometida a un proceso de limpieza topológica para eliminar lazos propios (\textit{self-loops}) y aristas redundantes. El resultado final fue una red conectada compuesta por **54 nodos** y **228 aristas**.


\subsection{Conversión a igraph y análisis topológico}

La red se transformó en un objeto \texttt{igraph} para profundizar en su estructura matemática. Las propiedades globales calculadas (Tabla \ref{tab:propiedades-red}) revelan una densidad de 0.1593. Este valor, junto con el grado medio de 8.44, indica que la red no es una estructura dispersa, sino un sistema cohesivo donde las proteínas tienden a formar grupos de interacción funcional significativa.

\begin{table}[htbp]
	\centering
	\caption{Propiedades topológicas globales de la red de Raynaud.}
	\label{tab:propiedades-red}
	\vspace{4pt}
	\renewcommand{\arraystretch}{1.25}
	\begin{tabular}{lc}
		\hline
		\textbf{Propiedad}            & \textbf{Valor} \\ \hline
		Número total de nodos         & 54             \\ 
		Número total de aristas       & 228            \\ 
		Grado medio                   & 8.44           \\ 
		Densidad de la red            & 0.1593         \\ 
		Diámetro de la red            & 5              \\ \hline
	\end{tabular}
\end{table}

Además de las métricas globales, se evaluó la importancia individual de los nodos mediante medidas de centralidad. El análisis de grado (\textit{degree}) y de intermediación (\textit{betweenness}) permitió identificar a los actores principales del sistema (Tabla \ref{tab:top_nodes}). 

Se observó que el gen \textbf{IRF5} actúa como el principal \textit{hub} de la red por conectividad directa. Sin embargo, el gen \textbf{IFIH1}, pese a tener menor grado, mostró el mayor valor de intermediación (0.178), lo que sugiere que actúa como un "cuello de botella" o puente crítico para el flujo de información entre diferentes módulos funcionales.







