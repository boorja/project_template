\section{Resultados}

\subsection{Obtención de los genes asociados al fenotipo (HPO)}

El estudio comenzó con la identificación de la base genética asociada al fenotipo clínico. A partir del término \textbf{Raynaud Phenomenon (HPO:0030880)}, se realizó una recuperación programática mediante la API oficial de la \textit{Human Phenotype Ontology}. Tras el procesamiento de la respuesta JSON y la depuración de duplicados, se consolidó un conjunto inicial de genes candidatos que constituyó el punto de partida biológico para las fases posteriores del análisis topológico.

\subsection{Construcción y Topología de la Red}

Los genes recuperados se mapearon a sus identificadores de proteína en la base de datos \textbf{STRINGdb} (versión 12.0) \cite{Szklarczyk2025}. Utilizando un umbral de confianza (\textit{combined score}) de 700, se construyó la red de interacción proteína-proteína (PPI). Tras el proceso de filtrado topológico para eliminar nodos desconectados, la red final quedó constituida por \textbf{39 nodos} y \textbf{67 aristas} (Figura \ref{fig:Network_Blobs}).

El análisis de las métricas globales (Tabla \ref{tab:global_stats}) revela características típicas de redes biológicas \textit{scale-free}. La baja densidad ($\rho = 0.09$) indica una conectividad dispersa, mientras que la elevada modularidad ($Q = 0.55$) evidencia una clara organización en comunidades funcionales. El coeficiente de clustering ($C = 0.43$), notablemente superior al esperado en una red aleatoria equivalente, confirma la presencia de triángulos funcionales y nichos locales altamente interconectados. El diámetro de 10 y la longitud de camino media de 3.68 sugieren una red con estructura de ``mundo pequeño'', donde la información biológica puede propagarse eficientemente entre módulos distantes.

\begin{table}[h!]
	\centering
	\caption{Estadísticas Globales de la Red PPI}
	\label{tab:global_stats}
	\begin{tabular}{lcc}
		\toprule
		\textbf{Métrica} & \textbf{Valor} & \textbf{Interpretación} \\
		\midrule
		Nodos Totales & 39 & Genes en la red \\
		Aristas Totales & 67 & Interacciones validadas \\
		Densidad ($\rho$) & 0.090 & Red dispersa (típico biológico) \\
		Grado Medio ($\bar{k}$) & 3.44 & $\sim$3-4 interacciones/gen \\
		Longitud de Camino ($\bar{L}$) & 3.68 & Comunicación eficiente \\
		Diámetro ($d$) & 10 & Máxima separación \\
		Coef. Clustering ($C$) & 0.427 & Alta agrupación local \\
		Modularidad ($Q$) & 0.549 & Estructura comunitaria fuerte \\
		\bottomrule
	\end{tabular}
\end{table}

\begin{figure}[h!]
	\centering
	\includegraphics[width=0.85\textwidth]{figures/Clusters_Blobs.png}
	\caption{Detección de comunidades mediante el algoritmo de Louvain. Se identificaron 7 clusters funcionales (diferenciados por color). Los \textit{blobs} delimitan las fronteras de cada comunidad, revelando la arquitectura modular de la red.}
	\label{fig:Network_Blobs}
\end{figure}

\subsection{Análisis de Centralidad: Identificación de Hubs y Cuellos de Botella}

El análisis de centralidad permitió identificar los nodos críticos que sostienen la arquitectura de la red (Tabla \ref{tab:nodes_info}). Se distinguen dos tipos de nodos funcionalmente relevantes:

\textbf{Hubs de conectividad:} El factor de transcripción \textbf{\textit{IRF5}} emergió como el principal \textit{hub} de la red ($k=10$, betweenness $= 0.344$), seguido de \textbf{\textit{IL10}} ($k=9$). Estos genes actúan como ``superconectores'' que integran múltiples vías de señalización inmune.

\textbf{Cuellos de botella (\textit{bottlenecks}):} De forma notable, \textbf{\textit{IFIH1}} ($k=4$, betweenness $= 0.306$) y \textbf{\textit{ADAR}} ($k=5$, betweenness $= 0.239$) presentan una betweenness desproporcionadamente alta respecto a su grado (Tabla \ref{tab:nodes_info}). Estos genes actúan como ``puentes'' críticos entre módulos funcionales, controlando el flujo de información entre la detección de ácidos nucleicos (Cluster 3) y la respuesta inmune efectora (Clusters 1 y 5), como puede observarse en la Figura \ref{fig:Red_Final}. Su posición topológica los convierte en dianas terapéuticas de especial interés.

\begin{table}[h!]
	\centering
	\caption{Genes con mayor centralidad en la red. Se destacan tanto los \textit{hubs} (alto grado) como los \textit{bottlenecks} (alta betweenness relativa al grado).}
	\label{tab:nodes_info}
	\begin{tabular}{lcccl}
		\toprule
		\textbf{Gen} & \textbf{Cluster} & \textbf{Grado} & \textbf{Betweenness} & \textbf{Rol Funcional} \\
		\midrule
		\textit{IRF5} & 4 & 10 & 0.344 & Hub: Activación inmune \\
		\textit{IL10} & 5 & 9 & 0.168 & Hub: Regulación anti-inflamatoria \\
		\textit{IFIH1} & 1 & 4 & 0.306 & Bottleneck: Sensor viral (MDA5) \\
		\textit{ADAR} & 3 & 5 & 0.239 & Bottleneck: Edición de RNA \\
		\textit{IRAK1} & 6 & 4 & 0.171 & Bottleneck: Señalización TLR \\
		\textit{TLR7} & 1 & 5 & 0.142 & Detección de ssRNA \\
		\textit{ITGAM} & 1 & 5 & 0.137 & Adhesión/Fagocitosis \\
		\textit{HLA-DRB1} & 5 & 5 & 0.104 & Presentación antigénica \\
		\bottomrule
	\end{tabular}
\end{table}

\subsection{Análisis Modular y Enriquecimiento Funcional}

El algoritmo de Louvain identificó \textbf{7 comunidades} con funciones biológicas diferenciadas, como se observa en la Figura \ref{fig:Network_Blobs}. El análisis de enriquecimiento de Gene Ontology (GO) reveló las siguientes especializaciones funcionales:

\subsubsection{Cluster 1: Inmunidad Innata y Receptores de Superficie}

Este módulo agrupa genes implicados en la respuesta inmune innata mediada por receptores de superficie (\textit{FCGR2A, FCGR2B, CR2, ITGAM, TLR7, IFIH1, RNF125}). El enriquecimiento funcional (Figura \ref{fig:Enrichment_C1}) destaca:

\begin{itemize}
	\item \textbf{Señalización de receptores Fc} (GO:0002768, $p_{adj} = 3.0 \times 10^{-4}$): Los receptores Fc$\gamma$ median la citotoxicidad celular dependiente de anticuerpos (ADCC) y la fagocitosis de inmunocomplejos.
	\item \textbf{Regulación de respuesta inmune efectora} (GO:0002697): Control de la activación leucocitaria.
	\item \textbf{Producción de IFN tipo I} (GO:0032479): Conexión con la vía del interferón a través de \textit{TLR7} e \textit{IFIH1}.
\end{itemize}

\subsubsection{Cluster 3: Interferonopatía y Metabolismo de Ácidos Nucleicos}

Este cluster representa el núcleo de la respuesta antiviral intrínseca (\textit{RNASEH2A/B/C, TREX1, ADAR, SAMHD1, STING1}). Los términos enriquecidos (Figura \ref{fig:Enrichment_C3}) apuntan directamente al eje de las \textbf{interferonopatías tipo I}:

\begin{itemize}
	\item \textbf{Reparación de errores de apareamiento} (GO:0006298, $p_{adj} = 1.1 \times 10^{-7}$): El complejo RNaseH2 degrada híbridos RNA:DNA durante la replicación.
	\item \textbf{Señalización de IFN tipo I} (GO:0060337, $p_{adj} = 1.1 \times 10^{-6}$): Activación sostenida del programa transcripcional de interferón.
	\item \textbf{Vía cGAS-STING} (GO:0140896): Detección de DNA citosólico aberrante que desencadena inflamación crónica.
\end{itemize}

Este módulo vincula el Fenómeno de Raynaud con el espectro de las interferonopatías (Síndrome de Aicardi-Goutières), donde mutaciones en estos genes causan acumulación de ácidos nucleicos endógenos que mimetizan infección viral.

\subsubsection{Cluster 4: Activación de Células B y Señalización de Receptores}

Liderado por el hub \textit{IRF5}, junto con \textit{BANK1, PTPN22, BLK, TNFSF4, TNFAIP3} (Figura \ref{fig:Enrichment_C4}):

\begin{itemize}
	\item \textbf{Respuesta a moléculas bacterianas} (GO:0002237, $p_{adj} = 9.2 \times 10^{-5}$): Activación de vías inflamatorias por PAMPs.
	\item \textbf{Producción de IL-6} (GO:0032635): Citoquina pro-inflamatoria clave en autoinmunidad.
	\item \textbf{Señalización del receptor de células B} (GO:0050853): \textit{BLK}, \textit{BANK1} y \textit{PTPN22} modulan el umbral de activación de linfocitos B.
\end{itemize}

Este cluster conecta el Fenómeno de Raynaud con enfermedades autoinmunes sistémicas donde predomina la disfunción humoral (LES, Esclerosis Sistémica).

\subsubsection{Cluster 5: Regulación de la Respuesta Adaptativa}

Módulo centrado en la tolerancia inmunológica y regulación de linfocitos (\textit{IL10, HLA-DRB1, PDCD1, CTLA4, FCGR3B, C4A/C4B}). El análisis de enriquecimiento (Figura \ref{fig:Enrichment_C5}) reveló:

\begin{itemize}
	\item \textbf{Inmunidad mediada por linfocitos} (GO:0002449, $p_{adj} = 1.4 \times 10^{-7}$): Respuesta adaptativa humoral y celular.
	\item \textbf{Regulación negativa de activación de células T} (GO:0050868, $p_{adj} = 1.1 \times 10^{-5}$): \textit{PDCD1} (PD-1) y \textit{CTLA4} son \textit{checkpoints} inmunes que previenen la autorreactividad.
	\item \textbf{Aclaramiento de células apoptóticas} (GO:2000425): Los componentes del complemento \textit{C4A/C4B} facilitan la eliminación de debris celular.
\end{itemize}

\subsubsection{Cluster 6: Regulación Epigenética y Estructura Nuclear}

A diferencia de los módulos inmunológicos, este cluster (\textit{LMNA, LBR, ZMPSTE24, MECP2, IRAK1}) presenta una firma funcional distintiva (Figura \ref{fig:Enrichment_C6}):

\begin{itemize}
	\item \textbf{Regulación epigenética} (GO:0040029, $p_{adj} = 1.8 \times 10^{-4}$): Control de la expresión génica mediante modificaciones de cromatina.
	\item \textbf{Formación de heterocromatina} (GO:0031507): Silenciamiento génico y estabilidad del genoma.
	\item \textbf{Procesamiento de miRNA} (GO:1903799): Regulación post-transcripcional por \textit{MECP2} y \textit{ZMPSTE24}.
\end{itemize}

La presencia de \textit{LMNA} (lamina A/C) y \textit{ZMPSTE24} vincula este cluster con las laminopatías, síndromes con afectación vascular y envejecimiento prematuro que pueden cursar con fenómenos vasomotores similares al Raynaud.

\begin{figure}[h!]
	\centering
	\begin{minipage}{0.48\textwidth}
		\centering
		\includegraphics[width=\linewidth]{figures/Enrichment_Cluster_1.png}
		\caption{Cluster 1: Inmunidad innata y receptores de superficie celular.}
		\label{fig:Enrichment_C1}
	\end{minipage}\hfill
	\begin{minipage}{0.48\textwidth}
		\centering
		\includegraphics[width=\linewidth]{figures/Enrichment_Cluster_3.png}
		\caption{Cluster 3: Señalización de IFN tipo I e interferonopatías.}
		\label{fig:Enrichment_C3}
	\end{minipage}
\end{figure}

\begin{figure}[h!]
	\centering
	\begin{minipage}{0.48\textwidth}
		\centering
		\includegraphics[width=\linewidth]{figures/Enrichment_Cluster_4.png}
		\caption{Cluster 4: Activación de células B y respuesta inflamatoria.}
		\label{fig:Enrichment_C4}
	\end{minipage}\hfill
	\begin{minipage}{0.48\textwidth}
		\centering
		\includegraphics[width=\linewidth]{figures/Enrichment_Cluster_5.png}
		\caption{Cluster 5: Regulación de la respuesta inmune adaptativa.}
		\label{fig:Enrichment_C5}
	\end{minipage}
\end{figure}

\begin{figure}[h!]
	\centering
	\includegraphics[width=0.6\textwidth]{figures/Enrichment_Cluster_6.png}
	\caption{Cluster 6: Regulación epigenética y estructura nuclear.}
	\label{fig:Enrichment_C6}
\end{figure}

\subsection{Integración: Arquitectura Funcional de la Red}

La visualización integrada de la red (Figura \ref{fig:Red_Final}) permite apreciar la convergencia de los distintos módulos funcionales. Los \textit{bottlenecks} identificados (\textit{IFIH1, ADAR, IRAK1}) actúan como nodos de integración que conectan la detección de ácidos nucleicos aberrantes (Cluster 3) con la respuesta inmune innata (Cluster 1) y la activación de células B (Cluster 4). Esta arquitectura sugiere que el Fenómeno de Raynaud, más allá de un trastorno vasomotor aislado, puede representar una manifestación periférica de desregulación inmune sistémica con componentes de interferonopatía y autoinmunidad humoral.

\begin{figure}[h!]
	\centering
	\includegraphics[width=0.95\textwidth]{figures/Red_Raynaud.png}
	\caption{Red de interacción génica final. El tamaño de los nodos es proporcional a su grado de conectividad. Los colores representan los 7 clusters funcionales. Nótese la posición central de \textit{IRF5} e \textit{IL10} como hubs integradores, y el rol de \textit{IFIH1} y \textit{ADAR} como puentes entre módulos.}
	\label{fig:Red_Final}
\end{figure}