\section{Materiales y métodos}

\subsection{Materiales}

\subsubsection{Bases de Datos Biológicas}
\begin{itemize}
	\item \textbf{Human Phenotype Ontology (HPO):} Fuente principal para la obtención de genes asociados a fenotipos clínicos. HPO utiliza un vocabulario estandarizado que permite una vinculación sistemática y reproducible entre las características de una enfermedad y su base genética \cite{Groza2023, Khler2021}.
	
	\item \textbf{Base de Datos STRING:} Se empleó para construir la red de interacciones. Esta es una base de datos integral que recopila y pondera interacciones proteína-proteína (PPI) a partir de múltiples fuentes de evidencia (experimental, computacional, etc.), asignando a cada una un puntaje de confianza \cite{Szklarczyk2025}.
\end{itemize}

\subsubsection{Software y Paquetes de Análisis}
Todos los análisis se realizaron en el entorno de programación \textbf{R} (v4.5.2 o superior), utilizando los siguientes paquetes:
\begin{itemize}
	\item \texttt{httr} (vx.x.x): Para la comunicación con la API de HPO.
	\item \texttt{jsonlite} (vx.x.x): Para el procesamiento de los datos en formato JSON obtenidos de la API.
	\item \texttt{STRINGdb} (vx.x.x): Para consultar la base de datos STRING y construir la red de PPI \cite{Szklarczyk2019}.
	\item \texttt{igraph} (vx.x.x): Herramienta central para la modelación, visualización y cálculo de propiedades topológicas de la red \cite{Csardi2006}.
	\item \texttt{clusterProfiler} (vx.x.x): Paquete de Bioconductor para la ejecución de análisis de enriquecimiento funcional \cite{Wu2021}.
\end{itemize}
