\section{Materiales y métodos}

\subsection{Materiales}

\subsubsection{Bases de datos biológicas}
\begin{itemize}
	\item \textbf{Human Phenotype Ontology (HPO):} Fuente principal para la obtención de genes asociados a fenotipos clínicos. HPO utiliza un vocabulario estandarizado que permite una vinculación sistemática y reproducible entre las características de una enfermedad y su base genética \cite{Groza2023, Khler2021}.
	
	\item \textbf{Base de Datos STRING:} Se empleó para construir la red de interacciones. Esta es una base de datos integral que recopila y pondera interacciones proteína-proteína (PPI) a partir de múltiples fuentes de evidencia (experimental, computacional, etc.), asignando a cada una un puntaje de confianza \cite{Szklarczyk2025}.
\end{itemize}

\subsubsection{Software y Paquetes de Análisis}
Todos los análisis se realizaron en el entorno de programación \textbf{R} (v4.5.2 o superior), utilizando los siguientes paquetes:
\begin{itemize}
	\item \texttt{httr} (vx.x.x): Para la comunicación con la API de HPO.
	\item \texttt{jsonlite} (vx.x.x): Para el procesamiento de los datos en formato JSON obtenidos de la API.
	\item \texttt{STRINGdb} (vx.x.x): Para consultar la base de datos STRING y construir la red de PPI \cite{Szklarczyk2019}.
	\item \texttt{igraph} (vx.x.x): Herramienta central para la modelación, visualización y cálculo de propiedades topológicas de la red \cite{Csardi2006}.
	\item \texttt{clusterProfiler} (vx.x.x): Paquete de Bioconductor para la ejecución de análisis de enriquecimiento funcional \cite{Wu2021}.
\end{itemize}

\subsubsection{Algoritmos y Enfoques Estadísticos}
\begin{itemize}
	\item \textbf{Algoritmo de Detección de Comunidades de Louvain:} Es un método heurístico utilizado para identificar la estructura modular en redes complejas. El algoritmo optimiza iterativamente una métrica de modularidad ($Q$), que cuantifica la densidad de las conexiones dentro de las comunidades en comparación con las conexiones entre ellas. La modularidad se define como:
	\begin{equation}
		Q = \frac{1}{2m} \sum_{i,j} \left[ A_{ij} - \frac{k_i k_j}{2m} \right] \delta(c_i, c_j)
		\label{eq:louvain_modularity}
	\end{equation}
	Donde:
	\begin{itemize}
		\item $A_{ij}$ es un elemento de la matriz de adyacencia, que vale 1 si los nodos $i$ y $j$ están conectados y 0 en caso contrario.
		\item $k_i$ y $k_j$ son los grados (número de conexiones) de los nodos $i$ y $j$.
		\item $m$ es el número total de aristas en la red.
		\item $\frac{k_i k_j}{2m}$ representa la probabilidad esperada de que exista una arista entre $i$ y $j$ en una red aleatoria con la misma distribución de grados.
		\item $c_i$ y $c_j$ son las comunidades a las que pertenecen los nodos $i$ y $j$.
		\item $\delta(c_i, c_j)$ es la función delta de Kronecker, que vale 1 si los nodos están en la misma comunidad ($c_i = c_j$) y 0 en caso contrario.
	\end{itemize}
	El algoritmo busca la partición de la red que maximiza el valor de $Q$, revelando subgrupos de nodos densamente conectados que se postula que comparten funciones biológicas.
	
	\item \textbf{Análisis de Sobrerrepresentación (ORA):} Es un enfoque estadístico para determinar si un conjunto de genes de interés está significativamente enriquecido en funciones o vías biológicas predefinidas. El método se basa en la prueba hipergeométrica para calcular el p-valor, que es la probabilidad de observar una superposición igual o mayor a la encontrada por puro azar. La fórmula es:
	\begin{equation}
		P(X \ge k) = \sum_{i=k}^{\min(n,K)} \frac{\binom{K}{i} \binom{N-K}{n-i}}{\binom{N}{n}}
		\label{eq:ora_hypergeometric}
	\end{equation}
	Donde:
	\begin{itemize}
		\item $N$ es el número total de genes en el genoma de fondo (background).
		\item $K$ es el número total de genes asociados al término funcional en estudio dentro del fondo.
		\item $n$ es el número de genes en el conjunto de interés (ej. genes sobreexpresados).
		\item $k$ es el número de genes en el conjunto de interés que también están asociados al término funcional.
	\end{itemize}
	Un p-valor bajo sugiere que la sobrerrepresentación observada no es casual, sino biológicamente significativa.
	
	\item \textbf{Corrección de Benjamini-Hochberg (BH):} Al realizar miles de pruebas estadísticas simultáneamente (una por cada término funcional), la probabilidad de obtener falsos positivos (errores de tipo I) se incrementa. El método de Benjamini-Hochberg controla la Tasa de Falso Descubrimiento (FDR), que es la proporción esperada de descubrimientos incorrectos. El procedimiento ordena los p-valores de menor a mayor ($p_{(1)} \le p_{(2)} \le \dots \le p_{(m)}$) y encuentra el mayor $k$ tal que:
	\begin{equation}
		p_{(k)} \le \frac{k}{m} \alpha
		\label{eq:bh_fdr}
	\end{equation}
	Donde:
	\begin{itemize}
		\item $p_{(k)}$ es el k-ésimo p-valor más pequeño de la lista ordenada.
		\item $m$ es el número total de pruebas realizadas (el número total de términos GO evaluados).
		\item $\alpha$ es el nivel de FDR que se desea controlar (comúnmente 0.05).
	\end{itemize}
	Todas las hipótesis nulas correspondientes a los p-valores $p_{(1)}, \dots, p_{(k)}$ se rechazan, considerándose significativas. Este ajuste es menos conservador que la corrección de Bonferroni y es ampliamente utilizado en genómica \cite{Benjamini1995}.
\end{itemize}

\subsection{Métodos}

\subsubsection{Adquisición de Datos y Construcción de la Red}
El conjunto inicial de genes se obtuvo mediante una consulta a la API de la \textbf{Human Phenotype Ontology (HPO)} utilizando el término "Raynaud phenomenon" (ID: \textbf{HP:0030881}). La lista de símbolos de genes resultante se utilizó como entrada para construir una red de interacción proteína-proteína (PPI) para \textit{Homo sapiens} a partir de la base de datos \textbf{STRING}. Se aplicó un filtro de alta confianza, reteniendo únicamente las interacciones con un \textbf{puntaje combinado superior a 0.800}. La red final se representó como un grafo no dirigido.

\subsubsection{Análisis Estructural y de Comunidades}
Las propiedades topológicas de la red fueron analizadas con el paquete \texttt{igraph}. Se calcularon métricas globales (número de nodos y aristas, densidad) y locales (grado promedio, centralidad de cercanía). Posteriormente, se aplicó el \textbf{algoritmo de Louvain} para particionar la red en módulos funcionales basándose en su estructura de conectividad.


\subsubsection{Análisis de Enriquecimiento Funcional}
Cada módulo identificado en el paso anterior fue sometido a un \textbf{Análisis de Sobrerrepresentación (ORA)} utilizando el paquete \texttt{clusterProfiler}. Se evaluó el enriquecimiento de términos de la ontología de \textbf{Proceso Biológico (BP)} de Gene Ontology (GO). Los p-valores resultantes se ajustaron mediante el método de \textbf{Benjamini-Hochberg}, y se consideraron como estadísticamente significativos aquellos términos con un \textbf{p-valor ajustado inferior a 0.05}.


