\section{Introducción}

El fenómeno de Raynaud (FR) es una condición que afecta a millones de personas en todo el mundo, caracterizada por episodios en los que los dedos se tornan pálidos, cianóticos y dolorosos ante la exposición al frío. A pesar de su frecuencia, sigue siendo un trastorno subdiagnosticado. Este fenómeno vasoespástico multifactorial se define por la constricción transitoria, recurrente y reversible de los vasos sanguíneos periféricos \cite{Nawaz2022, Herrick2012}. Clínicamente, se reconoce por un patrón trifásico de decoloración digital: primero palidez por isquemia, luego cianosis por la falta de oxigenación, y finalmente eritema como resultado de la reperfusión. Se estima que afecta aproximadamente al 5\% de la población general, con una marcada predisposición en mujeres, alcanzando una proporción de hasta 9:1 \cite{Herrick2012, Musa2023, Garner2015, Ingegnoli2022}.  

Desde un punto de vista patofisiológico, el FR refleja un delicado desequilibrio entre los mecanismos de vasoconstricción y vasodilatación que regulan la microcirculación digital. En este proceso convergen alteraciones endoteliales, del músculo liso vascular y de la modulación simpática \cite{Herrick2005, Flavahan2015}. A nivel celular, el punto de partida suele ser una disfunción endotelial que compromete la liberación de vasodilatadores esenciales como el óxido nítrico y la prostaciclina, al mismo tiempo que aumenta la producción de endotelina-1, un potente vasoconstrictor \cite{Flavahan2008, Blann1993, Freedman1999}. Este desequilibrio químico no se limita al endotelio: las células del músculo liso vascular responden con una hiperreactividad vasoconstrictora exagerada y una proliferación intimal progresiva \cite{Fardoun2016, Cooke2004}. A ello se suma la participación del sistema nervioso simpático, que amplifica el fenómeno mediante la liberación sostenida de norepinefrina y neuropéptidos vasoconstrictores. En el nivel molecular, destaca la implicación de los receptores adrenérgicos $\alpha$2C, los cuales, en respuesta al frío, se movilizan desde el retículo endoplasmático hacia la membrana celular, provocando así el vasoespasmo característico del trastorno \cite{Fardoun2016, Flavahan2008, Chotani2000}.  

Durante los últimos años, los avances en genómica han permitido esclarecer parte de la base genética del fenómeno de Raynaud. Un estudio de asociación del genoma completo (GWAS) de gran escala, que analizó más de cinco mil casos de FR y cerca de medio millón de controles, identificó por primera vez tres regiones genómicas significativamente asociadas al trastorno ($p < 5 \times 10^{-8}$), entre las que destacan los genes \textit{ADRA2A} e \textit{IRX1} como loci de susceptibilidad primarios \cite{Hartmann2023}. Investigaciones posteriores ampliaron este panorama, identificando variantes adicionales en genes implicados en el control del tono vascular, como \textit{NOS3} y \textit{ACVR2A}, así como en genes del sistema inmunitario, incluidos \textit{HLA} y \textit{NOS1}, este último con polimorfismos que modulan su expresión en tejido cutáneo \cite{deAlmeidaTervi2024, Smolina2018, Hughes2017}. La heredabilidad estimada en 7.7\% a partir de SNPs confirma la existencia de una contribución genética significativa a la patogénesis del FR \cite{Hartmann2023}.  

La complejidad biológica y clínica de este fenómeno hace que su estudio requiera una aproximación integradora. En este sentido, la biología de sistemas ofrece un marco conceptual y metodológico ideal para analizar el FR desde múltiples niveles de organización, articulando las redes de interacción que los vinculan. Una herramienta fundamental en este enfoque es la \textit{Human Phenotype Ontology} (HPO), que proporciona un lenguaje estandarizado para describir y vincular los fenotipos clínicos con sus bases genéticas mediante análisis computacionales reproducibles \cite{Khler2021, Robinson2008, Groza2023}. De forma complementaria, el análisis de redes de interacción proteína-proteína, a través de bases de datos como STRING, permite visualizar y modelar las relaciones funcionales entre los genes y las proteínas implicadas en el trastorno \cite{Szklarczyk2025, Orchard2014}. Este tipo de aproximaciones integradoras resulta esencial para identificar módulos funcionales y vías de señalización alteradas, así como posibles dianas terapéuticas \cite{Szklarczyk2025, Naylor2010, Fischer2025, Ideker2011}.  

En este contexto, el presente trabajo se propone integrar el conocimiento fenotípico estandarizado del HPO con el análisis de redes de interacción proteica, con el fin de esclarecer la arquitectura molecular subyacente al fenómeno de Raynaud. A través de este enfoque de biología de sistemas, se busca no solo confirmar la participación de genes ya conocidos, sino también descubrir nuevas rutas biológicas y módulos funcionales que contribuyan a comprender mejor la patogénesis de este complejo trastorno vasospástico, sentando así las bases para futuras investigaciones orientadas a su tratamiento \cite{Naylor2010, Fischer2025, Barabsi2011}.