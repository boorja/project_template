\section{Introducción}

Millones de personas experimentan episodios de dedos pálidos, cianóticos y dolorosos con el frío, un fenómeno frecuente pero a menudo infradiagnosticado. El fenómeno de Raynaud (FR) constituye un trastorno vasospástico multifactorial caracterizado por la constricción transitoria, recurrente y reversible de los vasos sanguíneos periféricos \cite{Nawaz2022, Herrick2012}. Clínicamente, se manifiesta mediante un patrón trifásico de decoloración en los dedos: palidez inicial por isquemia, cianosis por falta de oxigenación y, finalmente, eritema durante la reperfusión. Este trastorno afecta alrededor del 5\% de la población general y muestra una marcada predisposición femenina, con una relación de hasta 9:1 \cite{Medscape2024, Musa2023, Garner2015, Ingegnoli2022}.  

La patofisiología del fenómeno de Raynaud refleja un desequilibrio entre vasoconstricción y vasodilatación en la microcirculación digital, donde convergen alteraciones endoteliales, del músculo liso vascular y de la modulación simpática \cite{Herrick2005, Flavahan2015}. A nivel celular, este proceso se inicia con una disfuncción endotelial que compromete la producción de vasodilatadores clave como el óxido nítrico y la prostaciclina, mientras incrementa la liberación de endotelina-1 \cite{Flavahan2008, Blann1993, Freedman1999}. Este desequilibrio bioquímico no permanece confinado al endotelio: las células del músculo liso vascular responden a estas señales alteradas con una hiperreactividad vasoconstrictora exagerada y proliferación intimal progresiva \cite{Fardoun2016, Cooke2004}. A nivel molecular, el sistema nervioso simpático perpetúa y amplifica este ciclo mediante la liberación sostenida de norepinefrina y neuropéptidos vasoconstrictores. El mecanismo molecular central involucra a los receptores adrenérgicos $\alpha$2C, que en respuesta al frío experimentan una translocación desde el retículo endoplasmático hacia la membrana celular, contribuyendo así al vasoespasmo característico del trastorno \cite{Fardoun2016, Flavahan2008, Chotani2000}. 

Recientes avances en genómica han revelado importantes hallazgos sobre la base genética del fenómeno de Raynaud. Un estudio de asociación del genoma completo (GWAS) de gran escala que incluyó 5.147 casos de FR y 439.294 controles identificó por primera vez tres regiones genómicas robustamente asociadas con el FR ($p < 5 \times 10^{-8}$), destacando particularmente \textit{ADRA2A} e \textit{IRX1} como genes de susceptibilidad primarios \cite{Hartmann2023}. Análisis posteriores han ampliado este panorama genético, identificando variantes adicionales en genes relacionados con el tono vascular como \textit{NOS3}, \textit{ACVR2A} y genes de función inmune como \textit{HLA}, además de polimorfismos en \textit{NOS1} asociados con la expresión génica en tejido cutáneo \cite{deAlmeidaTervi2024, Smolina2018, Hughes2017}. La heredabilidad basada en SNPs se estimó en 7.7\%, confirmando una contribución genética significativa al trastorno \cite{Hartmann2023, ofLondon2023}.

La biología de sistemas ofrece un marco idóneo para desentrañar la complejidad multiescala del FR, al integrar en un mismo análisis los mecanismos vasculares, neurales y genéticos y la red de interacciones que los conecta. En este contexto, la \textit{Human Phenotipe Ontology} (HPO) aporta un vocabulario estandarizado de anomalías clínicas que posibilita vincular fenotipos bien definidos con su base genética mediante análisis computacionales reproducibles \cite{Khler2021, Robinson2008, Groza2023}. Complementariamente, el análisis de redes de interacción proteína-proteína, utilizando bases de datos como STRING, permite modelar y visualizar las relaciones funcionales entre los genes asociados al fenotipo \cite{Szklarczyk2025, Orchard2014}. Esta aproximación integradora es clave para identificar módulos funcionales y vías de señalización alteradas, revelando así potenciales dianas terapéuticas \cite{Consortium2021, Naylor2010, Fischer2025, Ideker2011}.

En este contexto, el presente trabajo tiene como objetivo integrar el conocimiento fenotípico estandarizado del HPO con análisis de redes de interacción proteica para elucidar la arquitectura molecular del fenómeno de Raynaud. Mediante este enfoque de biología de sistemas, se busca no solo validar la implicación de genes conocidos, sino también identificar nuevas rutas biológicas y módulos funcionales que contribuyen a la patogénesis de este complejo trastorno vasospástico, sentando así las bases para futuras investigaciones terapéuticas \cite{Naylor2010, Fischer2025, Barabsi2011}.