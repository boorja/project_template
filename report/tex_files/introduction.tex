\section{Introducción}

Millones de personas experimentan episodios de dedos pálidos, cianóticos y dolorosos con el frío, un fenómeno frecuente pero a menudo infradiagnosticado. El fenómeno de Raynaud (FR) constituye un trastorno vasospástico multifactorial caracterizado por la constricción transitoria, recurrente y reversible de los vasos sanguíneos periféricos \cite{Nawaz2022}.  Clínicamente, se manifiesta mediante un patrón trifásico de decoloración en los dedos: palidez inicial por isquemia, cianosis por falta de oxigenación y, finalmente, eritema durante la reperfusión. Este trastorno afecta alrededor del 5\% de la población general y muestra una marcada predisposición femenina, con una relación de hasta 9:1 \cite{Medscape2024, Musa2023}.  

La patofisiología del fenómeno de Raynaud involucra una compleja interacción de mecanismos vasculares, neurales e intravasculares que alteran el equilibrio entre vasoconstricción y vasodilatación \cite{Herrick2005}. A nivel molecular, se han identificado tres mecanismos principales: anomalías en el flujo sanguíneo, constricción vascular y respuestas neurogénicas. El sistema nervioso simpático desempeña un papel crucial mediante la liberación de norepinefrina y neuropéptidos vasoconstrictores que actúan sobre el músculo liso arteriolar. De particular relevancia es el papel de los receptores adrenérgicos $\alpha$2C, cuya translocación desde el retículo endoplasmático hacia la membrana celular en respuesta al frío contribuye significativamente al vasoespasmo exagerado \cite{Fardoun2016, Flavahan2008}.La patofisiología del fenómeno de Raynaud refleja un desequilibrio entre vasoconstricción y vasodilatación en la microcirculación digital, donde convergen alteraciones endoteliales, del músculo liso vascular y de la modulación simpática \cite{Herrick2005}. A nivel celular, este proceso se inicia con una disfunción endotelial que compromete la producción de vasodilatadores clave como el óxido nítrico y la prostaciclina, mientras incrementa la liberación de endotelina-1 \cite{Flavahan2008, Blann1993}. Este desequilibrio bioquímico no permanece confinado al endotelio, las células del músculo liso vascular responden a estas señales alteradas con una hiperreactividad vasoconstrictora exagerada y proliferación intimal progresiva \cite{Fardoun2016}. A nivel molecular, el sistema nervioso simpático perpetúa y amplifica este ciclo mediante la liberación sostenida de norepinefrina y neuropéptidos vasoconstrictores. El mecanismo molecular central involucra a los receptores adrenérgicos $\alpha$2C, que en respuesta al frío experimentan una translocación desde el retículo endoplasmático hacia la membrana celular contribuyendo así al vasoespasmo característico del trastorno \cite{Fardoun2016, Flavahan2008}.

Recientes avances en genómica han revelado importantes hallazgos sobre la base genética del fenómeno de Raynaud. Un estudio de asociación del genoma completo (GWAS) de gran escala identificó por primera vez genes causales robustamente asociados con el FR, destacando particularmente \textbf{ADRA2A} e \textbf{IRX1} como genes de susceptibilidad \cite{Hartmann2023}. El gen \textbf{ADRA2A} codifica el receptor adrenérgico $\alpha$2A para la adrenalina, un receptor de estrés clásico que causa la contracción de pequeños vasos sanguíneos. Por otro lado, \textbf{IRX1} es un factor de transcripción que puede regular la capacidad de los vasos sanguíneos para dilatarse, y su sobreproducción puede activar genes que impiden la relajación normal de los vasos constrictos \cite{ofLondon2023}.

La aplicación de enfoques de biología de sistemas ofrece un marco ideal para descifrar esta complejidad. La Ontología de Fenotipos Humanos (HPO) proporciona una descripción estandarizada de las anomalías fenotípicas, permitiendo un análisis computacional robusto que vincula los síntomas clínicos a sus bases genéticas \cite{Khler2021, Robinson2008}. Complementariamente, el análisis de redes de interacción proteína-proteína, utilizando bases de datos como STRING, permite modelar y visualizar las relaciones funcionales entre los genes asociados al fenotipo \cite{Szklarczyk2025}. Esta aproximación integradora es clave para identificar módulos funcionales y vías de señalización alteradas, revelando así potenciales dianas terapéuticas \cite{Consortium}.

En este contexto, el presente trabajo tiene como objetivo integrar el conocimiento fenotípico estandarizado del HPO con análisis de redes de interacción proteica para elucidar la arquitectura molecular del fenómeno de Raynaud. Mediante este enfoque de biología de sistemas, se busca no solo validar la implicación de genes conocidos, sino también identificar nuevas vías biológicas y módulos funcionales que contribuyen a la patogénesis de este complejo trastorno vasospástico, sentando así las bases para futuras investigaciones terapéuticas \cite{Naylor2010, Fischer2025}.
